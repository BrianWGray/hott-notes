\documentclass[11pt]{article}
%\metadata{Vonnie III}{2099/99/99}

\usepackage{mathpartir}

% for url in bib entries
\usepackage{url}

% Package for customizing page layout
\usepackage[a4paper,scale=0.8]{geometry}

\setlength{\parskip}{0.1cm}
\setlength{\parindent}{0mm}

% package to customize three basic list environments: enumerate, itemize and description.
\usepackage{enumitem}
\setitemize{noitemsep, topsep=0pt, leftmargin=*}
\setenumerate{noitemsep, topsep=0pt, leftmargin=*}
\setdescription{noitemsep, topsep=0pt, leftmargin=*}

% names
\newcommand{\HoTT}{Homotopy Type Theory}

%%% This file can never be completed.
%%% If you need something but cannot find it,
%%% contact the TA Favonia!

%%%%%%%%%%%%%%%%%%%%%%%%%%%%%%%%%%%%%%%%
% Basic packages
%%%%%%%%%%%%%%%%%%%%%%%%%%%%%%%%%%%%%%%%
\usepackage{amsmath,amsthm,amssymb}
\usepackage{fancyhdr}
\usepackage{mathpartir}
\usepackage{xcolor}
\usepackage{hyperref}
\usepackage{xspace}
\usepackage{comment}
\usepackage{url} % for url in bib entries

%%%%%%%%%%%%%%%%%%%%%%%%%%%%%%%%%%%%%%%%
% Acronyms
%%%%%%%%%%%%%%%%%%%%%%%%%%%%%%%%%%%%%%%%
\usepackage[acronym, shortcuts]{glossaries}

\newacronym{HoTT}{HoTT}{homotopy type theory}
\newacronym{IPL}{IPL}{intuitionistic propositional logic}
\newacronym{TT}{TT}{intuitionistic type theory}
\newacronym{LEM}{LEM}{law of the excluded middle}
\newacronym{ITT}{ITT}{intensional type theory}
\newacronym{ETT}{ETT}{extensional type theory}
\newacronym{NNO}{NNO}{natural numbers object}

%%%%%%%%%%%%%%%%%%%%%%%%%%%%%%%%%%%%%%%%
% Fancy page style
%%%%%%%%%%%%%%%%%%%%%%%%%%%%%%%%%%%%%%%%
\pagestyle{fancy}
\newcommand{\metadata}[2]{
  \lhead{}
  \chead{}
  \rhead{\bfseries Homotopy Type Theory}
  \lfoot{#1}
  \cfoot{#2}
  \rfoot{\thepage}
}
\renewcommand{\headrulewidth}{0.4pt}
\renewcommand{\footrulewidth}{0.4pt}


\newcommand*{\vocab}[1]{\emph{#1}}
\newcommand*{\latin}[1]{\textit{#1}}

%%%%%%%%%%%%%%%%%%%%%%%%%%%%%%%%%%%%%%%%
% Customize list enviroonments
%%%%%%%%%%%%%%%%%%%%%%%%%%%%%%%%%%%%%%%%
% package to customize three basic list environments: enumerate, itemize and description.
\usepackage{enumitem}
\setitemize{noitemsep, topsep=0pt, leftmargin=*}
\setenumerate{noitemsep, topsep=0pt, leftmargin=*}
\setdescription{noitemsep, topsep=0pt, leftmargin=*}

%%%%%%%%%%%%%%%%%%%%%%%%%%%%%%%%%%%%%%%%
% Some really basic macros.
% (Lots of them were stolen from HoTT/Book.)
% See macros.tex in HoTT/book.
%%%%%%%%%%%%%%%%%%%%%%%%%%%%%%%%%%%%%%%%
\newcommand*{\ctx}{\Gamma}
\newcommand{\entails}{\vdash}


\newcommand*{\judgmentfont}[1]{{\normalfont\sffamily #1}}
\newcommand*{\postfixjudgment}[1]{%
  \relax\ifnum\lastnodetype>0\mskip\medmuskip\fi
  \text{\judgmentfont{#1}}%
}
\newcommand*{\prop}{\postfixjudgment{prop}}
\newcommand*{\true}{\postfixjudgment{true}}
\newcommand*{\type}{\postfixjudgment{type}}
\newcommand*{\context}{\postfixjudgment{ctx}}


\newcommand*{\truth}{\top}
\newcommand*{\conj}{\wedge}
\newcommand*{\disj}{\vee}
\newcommand*{\falsehood}{\bot}
\newcommand*{\imp}{\supset}


%%% Judgmental equality
\newcommand{\jdeq}{\equiv}
%%% Definition
\newcommand{\defeq}{\vcentcolon\equiv}
%%% Binary sums
\newcommand{\inlsym}{{\mathsf{inl}}}
\newcommand{\inrsym}{{\mathsf{inr}}}
\newcommand{\inl}{\ensuremath\inlsym\xspace}
\newcommand{\inr}{\ensuremath\inrsym\xspace}
%%% Booleans
\newcommand{\ttsym}{{\mathsf{tt}}}
\newcommand{\ffsym}{{\mathsf{ff}}}
%\newcommand{\ttrue}{\ensuremath\ttsym\xspace}
%\newcommand{\ffalse}{\ensuremath\ffsym\xspace}
%%% Pairs
\newcommand{\pair}{\ensuremath{\mathsf{pair}}\xspace}
\newcommand{\tuple}[2]{(#1,#2)}
\newcommand{\proj}[1]{\ensuremath{\mathsf{pr}_{#1}}\xspace}
\newcommand{\fst}{\ensuremath{\proj1}\xspace}
\newcommand{\snd}{\ensuremath{\proj2}\xspace}
%%% Path concatenation
\newcommand{\concat}{%
  \mathchoice{\mathbin{\raisebox{0.5ex}{$\displaystyle\centerdot$}}}%
  {\mathbin{\raisebox{0.5ex}{$\centerdot$}}}%
  {\mathbin{\raisebox{0.25ex}{$\scriptstyle\,\centerdot\,$}}}%
  {\mathbin{\raisebox{0.1ex}{$\scriptscriptstyle\,\centerdot\,$}}}
}
%%% Transport (covariant)
\newcommand{\trans}[2]{\ensuremath{{#1}_{*}\mathopen{}\left({#2}\right)\mathclose{}}\xspace}
% Natural numbers objects
\newcommand{\Nat}{\mathsf{Nat}}
\newcommand{\rec}{\ensuremath{\mathsf{rec}}\xspace}
% Sequence
\newcommand{\Seq}{\ensuremath{\mathsf{Seq}}\xspace}
% Identity type
\newcommand{\Id}[1]{\ensuremath{\mathsf{Id}_{#1}}\xspace}
% Reflection
\newcommand{\refl}[1]{\ensuremath{\mathsf{refl}_{#1}}\xspace}

% fst,snd,case,id
\renewcommand*{\fst}{\textsf{fst}}
\renewcommand*{\snd}{\textsf{snd}}
\DeclareMathOperator{\case}{\textsf{case}}
\DeclareMathOperator{\caseif}{\textsf{if}}
\DeclareMathOperator{\casesplit}{\textsf{split}}
\DeclareMathOperator{\ttrue}{\textsf{tt}\xspace}
\DeclareMathOperator{\ffalse}{\textsf{ff}\xspace}
\newcommand*{\id}{\textsf{id}}


\begin{document}

\title{15-819 Homotopy Type Theory\\ Lecture Notes}
\author{Henry DeYoung and Stephanie Balzer}
\date{September 9 and 11, 2013}

\maketitle

\section{Contents}\label{sec:contents}

These notes summarize the lectures on \HoTT\ (HoTT) given by Professor Robert Harper on
September 9 and 11, 2013, at CMU.  They start by providing a introduction to HoTT, capturing
its main ideas and its connection to other related type theories.  Then they present
Intuitionistic Propositional Logic (IPL), giving both an inference-rule-based formulation as
well an order-theoretic formulation.  The notes conclude with a brief summary.

% [COMMENT_SB:] How does IPL relate to HoTT?

\section{Introduction to \HoTT\ (HoTT)}\label{sec:intro}

\subsection{In a Nutshell}\label{subsec:hott_in_nutshell}

The notion of proof relevance requires us to draw a fine distinction between the proofs given
in a intuitionistic logic and the ones given in conventional mathematics.
Harper~\cite{Harper2012} calls the former \textbf{proofs} and the latter \textbf{formal
  proofs}.  Formal proofs arise from the application of the inductively defined rules in a
fixed formal system.  According to G\"{o}del's Incompleteness Theorem, however, we know that
there are certain propositions that are true for which there is no formal proof.  Thus, an
intuitionistic approach defeats the adoption of a formal proof since in intuitionism true
propositions are only those for which there exists a proof.

According to Harper~\cite{Harper2013a,Harper2013}, HoTT:

\begin{itemize}

\item is a nice way to phrase arguments in homotopy theory that avoids some of the technicalities
  in the classical proofs by treating spaces and paths \textbf{synthetically}, rather than
  \textbf{analytically}

\item is a good language for \textbf{mechanization} of mathematics that provides for the concise
  formulation of proofs in a form that can be verified by a computer

\item points the way towards a vast extension of the concept of computation that enables us to
  \textbf{compute with abstract geometric objects} such as spheres or toruses

\item is a new \textbf{foundation for mathematics} that subsumes set theory by generalizing
  types from mere sets to arbitrary infinity groupoids, sets being but particularly simple
  types (those with no non-trivial higher-dimensional structure).

\end{itemize}

HoTT's most distinctive feature is its \textbf{constructivity}.

\subsection{In Type Theory Context}\label{subsec:type_theory_context}

\section{Intuitionistic Propositional Logic (IPL)}\label{sec:ipl}

\subsection{Inference-Rule-Based Formulation}\label{subsec:ipl_rules}

% [COMMENT_SB:] Explain difference between introduction and elimination rules.  Mention
% Gentzen's rule of inversion.

\subsubsection{Negative Fragment}\label{subsubsec:ipl_negfrag}

\subsubsection{Positive Fragment}\label{subsubsec:ipl_posfrag}

\subsection{Order-Theoretic Formulation}\label{subsec:ipl_order}

\section{Summary}\label{sec:summary}

% \section{Example Formulae}

% This is a paragraph with no purpose other than
% occupying some space.

% \begin{mathpar}
%     \infer*[Right=Reflection]{
%       \G \entails \bot \to \bot
%     }{
%       \G \entails \top \to \top
%     }
%     \\
%     \infer*[]{
%     }{
%       \G \entails \fst(\trans{p}{x}) \concat z \jdeq y : A
%     }
% \end{mathpar}

% The above rule is as insightful as this paragraph.

\bibliographystyle{plain}
\bibliography{hott_references}

\end{document}
