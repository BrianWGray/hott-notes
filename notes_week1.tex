% arara: pdflatex
% arara: bibtex
% arara: pdflatex
% arara: pdflatex
\documentclass[12pt]{article}

\usepackage{lmodern}
\usepackage{microtype}

% Package for customizing page layout
\usepackage[a4paper,scale=0.8]{geometry}

%%% This file can never be completed.
%%% If you need something but cannot find it,
%%% contact the TA Favonia!

%%%%%%%%%%%%%%%%%%%%%%%%%%%%%%%%%%%%%%%%
% Basic packages
%%%%%%%%%%%%%%%%%%%%%%%%%%%%%%%%%%%%%%%%
\usepackage{amsmath,amsthm,amssymb}
\usepackage{fancyhdr}
\usepackage{mathpartir}
\usepackage{xcolor}
\usepackage{hyperref}
\usepackage{xspace}
\usepackage{comment}
\usepackage{url} % for url in bib entries

%%%%%%%%%%%%%%%%%%%%%%%%%%%%%%%%%%%%%%%%
% Acronyms
%%%%%%%%%%%%%%%%%%%%%%%%%%%%%%%%%%%%%%%%
\usepackage[acronym, shortcuts]{glossaries}

\newacronym{HoTT}{HoTT}{homotopy type theory}
\newacronym{IPL}{IPL}{intuitionistic propositional logic}
\newacronym{TT}{TT}{intuitionistic type theory}
\newacronym{LEM}{LEM}{law of the excluded middle}
\newacronym{ITT}{ITT}{intensional type theory}
\newacronym{ETT}{ETT}{extensional type theory}
\newacronym{NNO}{NNO}{natural numbers object}

%%%%%%%%%%%%%%%%%%%%%%%%%%%%%%%%%%%%%%%%
% Fancy page style
%%%%%%%%%%%%%%%%%%%%%%%%%%%%%%%%%%%%%%%%
\pagestyle{fancy}
\newcommand{\metadata}[2]{
  \lhead{}
  \chead{}
  \rhead{\bfseries Homotopy Type Theory}
  \lfoot{#1}
  \cfoot{#2}
  \rfoot{\thepage}
}
\renewcommand{\headrulewidth}{0.4pt}
\renewcommand{\footrulewidth}{0.4pt}


\newcommand*{\vocab}[1]{\emph{#1}}
\newcommand*{\latin}[1]{\textit{#1}}

%%%%%%%%%%%%%%%%%%%%%%%%%%%%%%%%%%%%%%%%
% Customize list enviroonments
%%%%%%%%%%%%%%%%%%%%%%%%%%%%%%%%%%%%%%%%
% package to customize three basic list environments: enumerate, itemize and description.
\usepackage{enumitem}
\setitemize{noitemsep, topsep=0pt, leftmargin=*}
\setenumerate{noitemsep, topsep=0pt, leftmargin=*}
\setdescription{noitemsep, topsep=0pt, leftmargin=*}

%%%%%%%%%%%%%%%%%%%%%%%%%%%%%%%%%%%%%%%%
% Some really basic macros.
% (Lots of them were stolen from HoTT/Book.)
% See macros.tex in HoTT/book.
%%%%%%%%%%%%%%%%%%%%%%%%%%%%%%%%%%%%%%%%
\newcommand*{\ctx}{\Gamma}
\newcommand{\entails}{\vdash}


\newcommand*{\judgmentfont}[1]{{\normalfont\sffamily #1}}
\newcommand*{\postfixjudgment}[1]{%
  \relax\ifnum\lastnodetype>0\mskip\medmuskip\fi
  \text{\judgmentfont{#1}}%
}
\newcommand*{\prop}{\postfixjudgment{prop}}
\newcommand*{\true}{\postfixjudgment{true}}
\newcommand*{\type}{\postfixjudgment{type}}
\newcommand*{\context}{\postfixjudgment{ctx}}


\newcommand*{\truth}{\top}
\newcommand*{\conj}{\wedge}
\newcommand*{\disj}{\vee}
\newcommand*{\falsehood}{\bot}
\newcommand*{\imp}{\supset}


%%% Judgmental equality
\newcommand{\jdeq}{\equiv}
%%% Definition
\newcommand{\defeq}{\vcentcolon\equiv}
%%% Binary sums
\newcommand{\inlsym}{{\mathsf{inl}}}
\newcommand{\inrsym}{{\mathsf{inr}}}
\newcommand{\inl}{\ensuremath\inlsym\xspace}
\newcommand{\inr}{\ensuremath\inrsym\xspace}
%%% Booleans
\newcommand{\ttsym}{{\mathsf{tt}}}
\newcommand{\ffsym}{{\mathsf{ff}}}
%\newcommand{\ttrue}{\ensuremath\ttsym\xspace}
%\newcommand{\ffalse}{\ensuremath\ffsym\xspace}
%%% Pairs
\newcommand{\pair}{\ensuremath{\mathsf{pair}}\xspace}
\newcommand{\tuple}[2]{(#1,#2)}
\newcommand{\proj}[1]{\ensuremath{\mathsf{pr}_{#1}}\xspace}
\newcommand{\fst}{\ensuremath{\proj1}\xspace}
\newcommand{\snd}{\ensuremath{\proj2}\xspace}
%%% Path concatenation
\newcommand{\concat}{%
  \mathchoice{\mathbin{\raisebox{0.5ex}{$\displaystyle\centerdot$}}}%
  {\mathbin{\raisebox{0.5ex}{$\centerdot$}}}%
  {\mathbin{\raisebox{0.25ex}{$\scriptstyle\,\centerdot\,$}}}%
  {\mathbin{\raisebox{0.1ex}{$\scriptscriptstyle\,\centerdot\,$}}}
}
%%% Transport (covariant)
\newcommand{\trans}[2]{\ensuremath{{#1}_{*}\mathopen{}\left({#2}\right)\mathclose{}}\xspace}
% Natural numbers objects
\newcommand{\Nat}{\mathsf{Nat}}
\newcommand{\rec}{\ensuremath{\mathsf{rec}}\xspace}
% Sequence
\newcommand{\Seq}{\ensuremath{\mathsf{Seq}}\xspace}
% Identity type
\newcommand{\Id}[1]{\ensuremath{\mathsf{Id}_{#1}}\xspace}
% Reflection
\newcommand{\refl}[1]{\ensuremath{\mathsf{refl}_{#1}}\xspace}

% fst,snd,case,id
\renewcommand*{\fst}{\textsf{fst}}
\renewcommand*{\snd}{\textsf{snd}}
\DeclareMathOperator{\case}{\textsf{case}}
\DeclareMathOperator{\caseif}{\textsf{if}}
\DeclareMathOperator{\casesplit}{\textsf{split}}
\DeclareMathOperator{\ttrue}{\textsf{tt}\xspace}
\DeclareMathOperator{\ffalse}{\textsf{ff}\xspace}
\newcommand*{\id}{\textsf{id}}


\usepackage{proof}
\usepackage{glossaries}

\metadata{Vonnie III}{2099/99/99}

% for url in bib entries
\usepackage{url}

\usepackage{parskip}
% \setlength{\parskip}{0.1cm}
% \setlength{\parindent}{0mm}

% package to customize three basic list environments: enumerate, itemize and description.
\usepackage{enumitem}
\setitemize{noitemsep, topsep=0pt, leftmargin=*}
\setenumerate{noitemsep, topsep=0pt, leftmargin=*}
\setdescription{noitemsep, topsep=0pt, leftmargin=*}

\begin{document}

\title{15-819 Homotopy Type Theory\\ Lecture Notes}
\author{Henry DeYoung and Stephanie Balzer}
\date{September 9 and 11, 2013}

\maketitle

% [COMMENT_SB:]

% > We should decide on capitalization of terms like Homotopy Type Theory and Intuitionistic
%   Propositional Logic.

% > Should we have a title for inference-rule based formulation of IPL?

% > Is it Brouwer's program, principle, or dictum?

\section{Contents}\label{sec:contents}

These notes summarize the lectures on Homotopy Type Theory (HoTT) given by Professor Robert
Harper on September 9 and 11, 2013, at CMU.  They start by providing a introduction to HoTT,
capturing its main ideas and its connection to other related type theories.  Then they present
Intuitionistic Propositional Logic (IPL), giving both an inference-rule-based formulation as
well an order-theoretic formulation.  The notes conclude with a brief summary.

% [COMMENT_SB:] How does IPL relate to HoTT?

\section{Introduction to Homotopy Type Theory}\label{sec:intro}

\newacronym{HoTT}{HoTT}{Homotopy Type Theory}%
%
\gls{HoTT} is the subject of a very active research community that gathered in 2012 at the Institute for Advanced
Study (IAS) to participate in the Univalent Foundations Program.  The results of the program
have been recently published in the HoTT Book~\cite{HoTTBook2013}.

\subsection{HoTT in a nutshell}\label{subsec:hott_in_nutshell}

\gls{HoTT} is based on Per Martin-L\"{o}f's Intuitionistic Type Theory, which provides a
foundation for \emph{intuitionistic mathematics} and which is an extension of \emph{Brouwer's
  program}.  Brouwer viewed mathematical reasoning as a human activity and mathematics as a
language for communicating mathematical concepts.  As a result, Brouwer perceives the ability
of executing a step-by-step procedure or algorithm for performing a \emph{construction} as a
fundamental human faculty.

Adopting Brouwer's constructive viewpoint, intuitionistic theories view proofs as the
fundamental forms of construction.  An intuitionistic (or constructive) theory and \gls{HoTT},
in particular, is thus said to be is \emph{proof relevant}.  For \gls{HoTT} proof relevance
means that proofs become mathematical objects~\cite{Harper2013}.  Proofs of equality correspond
to paths in a space.

The notion of proof relevance requires us to draw a fine distinction between the proofs given
in a intuitionistic logic and the ones given in conventional mathematics.
Harper~\cite{Harper2012} calls the former \emph{proofs} and the latter \emph{formal
  proofs}.  Formal proofs arise from the application of the inductively defined rules in a
fixed formal system.  According to G\"{o}del's Incompleteness Theorem, however, we know that
there are certain propositions that are true for which there is no formal proof.  Thus, an
intuitionistic approach defeats the adoption of a formal proof since in intuitionism true
propositions are only those for which there exists a proof.

According to Harper~\cite{Harper2013a,Harper2013}, HoTT:

\begin{itemize}

\item is a nice way to phrase arguments in homotopy theory that avoids some of the technicalities
  in the classical proofs by treating spaces and paths \emph{synthetically}, rather than
  \emph{analytically}

\item is a good language for \emph{mechanization} of mathematics that provides for the concise
  formulation of proofs in a form that can be verified by a computer

\item points the way towards a vast extension of the concept of computation that enables us to
  \emph{compute with abstract geometric objects} such as spheres or toruses

\item is a new \emph{foundation for mathematics} that subsumes set theory by generalizing
  types from mere sets to arbitrary infinity groupoids, sets being but particularly simple
  types (those with no non-trivial higher-dimensional structure).

\end{itemize}

HoTT's most distinctive feature is its \emph{constructivity}.

\subsection{HoTT in type theory context}\label{subsec:type_theory_context}

Extensional Type Theory (ETT).  There is the conception that ETT is ``wrong''~\cite{Harper2012}
because the typing judgment is not decidable.

\section{Intuitionistic propositional logic: ``Logic as if people matter''}\label{sec:ipl}

% [COMMENT_SB:] Explain difference between introduction and elimination rules.  Mention
% Gentzen's rule of inversion.

What is meant by \emph{intuitionistic} logic?  One might say its slogan is ``logic as if people matter'', alluding to Brouwer's principle that mathematics is a social process.

\newacronym{IPL}{IPL}{intuitionistic propositional logic}%
%
As advanced by Per Martin-L\"{o}f, a modern presentation of \gls{IPL} distinguishes the notions of \vocab{judgment} and \vocab{proposition}.
A judgment is something that may be known, whereas a proposition is something that sensibly be may the subject of a judgment.
For instance, 

The principle of harmony is that the introduction and elimination rules for a proposition $A$ fit together properly.  The elimination rules should be strong enough to deduce all information that was used to introduce $A$ (\vocab{local completeness}), but not so strong as to deduce information that might not have been used to introduce $A$ (\vocab{local soundness}).  In a later lecture, we will discuss harmony more precisely, but we can already give an informal treatment.

% Ultimately, we are interested in judging the truth, or provability, of a proposition.
% But first, 

Thus, in \gls{IPL}, the two most basic judgments are $A \prop$ and $A \true$:


\subsection{Negative fragment of \gls{IPL}}\label{sec:ipl-negative}

\subsubsection{Conjunction}\label{sec:conjunction}

One familiar group of propositions are the conjunctions.
If $A$ and $B$ are well-formed propositions, then so is their conjunction, which we write as $A \conj B$.
The formation rule for conjunction serves as evidence of the judgment $A \conj B \prop$, that $A \conj B$ is a well-formed proposition, provided that there is evidence of the judgments $A \prop$ and $B \prop$.
\begin{equation*}
  \infer[{\conj}F]{A \conj B \prop}{
    A \prop & B \prop}
\end{equation*}

We have yet to give meaning to conjunction, however; to do so, we must say how to introduce the judgment that $A \conj B$ is \emph{true}.
As the following rule shows, a verification of $A \conj B$ consists of a proof of $A \true$ paired with a proof of $B \true$.
\begin{equation*}
  \infer[{\conj}I]{A \conj B \true}{
    A \true & B \true}
\end{equation*}

What may we deduce from the knowledge that $A \conj B$ is true?
Because every proof of $A \conj B \true$ ultimately introduces that judgment from a pair of proofs of $A \true$ and $B \true$, we are justified in deducing $A \true$ and $B \true$ from any proof of $A \conj B \true$.
This leads to the elimination rules for conjunction.
\begin{mathpar}
  \infer[{\conj}E_1]{A \true}{
    A \conj B \true}
  \and
  \infer[{\conj}E_2]{B \true}{
    A \conj B \true}
\end{mathpar}

\paragraph{Harmony.}\label{sec:conj-harmony}
As previously mentioned, the principle of harmony says that the introduction and elimination rules fit together properly: the elimination rules are strong enough, but not too strong.

If we mistakenly omitted the ${\conj}E_2$ elimination rule, then there would be no way to extract the proof of $B \true$ that was used in introducing $A \conj B \true$---the elimination rules would too weak.

On the other hand, if we mistakenly wrote the ${\conj}I$ introduction rule as
\begin{equation*}
  \infer{A \conj B \true}{
    A \true} \,,
\end{equation*}
then there would be no proof of $B \true$ present to justify deducing $B \true$ with the ${\conj}E_2$ rule---the elimination rules would be too strong.

\subsubsection{Truth}\label{sec:truth}

Another familiar and simple proposition is \vocab{truth}, which we write as $\truth$.
Its formation rule serves as immediate evidence of the judgment $\truth \prop$, that $\truth$ is indeed a well-formed proposition.
\begin{equation*}
  \infer[{\truth}F]{\truth \prop}{
    }
\end{equation*}


\paragraph{Weakening.}\label{sec:weakening}
Suppose



\subsubsection{Implication}\label{sec:implication}

\subsubsection{Summary of the negative fragment of \gls{IPL}}\label{sec:summary-negative}

\subsection{Positive fragment of \gls{IPL}}\label{sec:positive}

\subsubsection{Disjunction}\label{sec:disjunction}

As for conjunction and implication, the disjunction, $A \disj B$, of $A$ and $B$ is a well-formed proposition if both $A$ and $B$ are themselves well-formed propositions.
\begin{equation*}
  \infer[{\disj}F]{A \disj B \prop}{
    A \prop & B \prop}
\end{equation*}

A disjunction $A \disj B$ may be introduced in either of two ways: $A \disj B$ is true if $A$ is true or if $B$ is true.
\begin{mathpar}
  \infer[{\disj}I_1]{A \disj B \true}{
    A \true}
  \and
  \infer[{\disj}I_2]{A \disj B \true}{
    B \true}
\end{mathpar}
To devise the elimination rule, consider what may we deduce from the knowledge that $A \disj B$ is true.
For $A \disj B$ to be true, it must have been ultimately introduced using one of the two introduction rules.
Therefore, either $A$ or $B$ is true (or possibly both).
The elimination rule allows us to reason by these two cases: If $C \true$ follows from $A \true$ and also follows from $B \true$, then $C$ is true in either case.
\begin{equation*}
  \infer[{\disj}E]{C \true}{
    A \disj B \true &
    A \true \entails C \true & B \true \entails C \true}
\end{equation*}

\subsubsection{Falsehood}\label{sec:falsehood}

The unit of disjunction is falsehood, the proposition that is trivially never true, which we write as $\falsehood$.  Its formation rule is immediate evidence that $\falsehood$ is a well-formed proposition.
\begin{equation*}
  \infer[{\falsehood}F]{\falsehood \prop}{
    }
\end{equation*}

Because $\falsehood$ should never be true, it has no introduction rule.
The elimination rule captures \latin{ex falso quodlibet}: from a proof of $\falsehood \true$, we may deduce that \emph{any} proposition $C$ is true because there is ultimately no way to introduce $\falsehood \true$.
\begin{equation*}
  \infer[{\falsehood}E]{C \true}{
    \falsehood \true}
\end{equation*}

Once again, the rules are in harmony.
The elimination rule is very strong, but remains justified due to the absence of any introduction rule for falsehood.

\paragraph{Nullary disjunction.}\label{sec:nullary-disjunction}
We previously noted that $\truth$ behaves as a nullary conjunction.
In the same way, $\falsehood$ behaves as a nullary disjunction.
For a binary disjunction, there are two introduction rules, ${\disj}I_1$ and ${\disj}I_2$, one for each of the two disjuncts; for falsehood, there are no introduction rules:
\begin{mathpar}
  \infer[{\disj}I_1]{A \disj B \true}{
    A \true}
  \and
  \infer[{\disj}I_2]{A \disj B \true}{
    B \true}
  \and
  \text{(no ${\falsehood}I$ rule)}
\end{mathpar}
Likewise, for a binary disjunction, there is one elimination rule with a premise for the disjunction and one premise for each of the disjuncts; for falsehood, there is one elimination rule with just a premise for falsehood:
\begin{mathpar}
  \infer[{\disj}E]{C \true}{
    A \disj B \true &
    A \true \entails C \true & B \true \entails C \true}
  \and
  \infer[{\falsehood}E]{C \true}{
    \falsehood \true}
\end{mathpar}

\section{Order-theoretic formulation of \gls{IPL}}\label{sec:ipl_order}

It is also possible to give an order-theoretic formulation of \gls{IPL} because entailment is a preorder (reflexive and transitive).
We define $A \leq B$ to hold exactly when $A \true \entails B \true$.

\subsection{Conjunction as meet}\label{sec:conjunction-as-meet}

The elimination rules for conjunction (along with reflexivity of entailment) ensure that $A \conj B \true \entails A \true$ and $A \conj B \true \entails B \true$.
Order-theoretically, this is expressed as the rules
\begin{mathpar}
  \infer{A \conj B \leq A}{
    }
  \and
  \infer{A \conj B \leq B}{
    }
\end{mathpar}
which say that $A \conj B$ is a lower bound of $A$ and $B$.

The introduction rule for conjunction ensures that $C \true \entails A \conj B \true$ if both $C \true \entails A \true$ and $C \true \entails B \true$.
Order-theoretically, this is expressed as the rule
\begin{equation*}
  \infer{C \leq A \conj B}{
    C \leq A & C \leq B}
\end{equation*}
which says that $A \conj B$ is as large as any lower bound of $A$ and $B$.
Taken together these rules show that $A \conj B$ is the greatest lower bound, or meet, of $A$ and $B$.

\subsection{Truth as greatest element}\label{sec:truth-as-greatest}

Truth $\truth$ should 

\subsection{Disjunction as join}\label{sec:disjunction-as-join}

\subsection{Falsehood as least element}\label{sec:falsehood-as-least}

\subsection{Implication as exponential}\label{sec:impl-as-expon}


\section{Summary}\label{sec:summary}

\bibliographystyle{plain}
\bibliography{hott_references}

\end{document}
