\documentclass[12pt]{article}
\usepackage[letterpaper]{geometry}

%%% This file can never be completed.
%%% If you need something but cannot find it,
%%% contact the TA Favonia!

%%%%%%%%%%%%%%%%%%%%%%%%%%%%%%%%%%%%%%%%
% Basic packages
%%%%%%%%%%%%%%%%%%%%%%%%%%%%%%%%%%%%%%%%
\usepackage{amsmath,amsthm,amssymb}
\usepackage{fancyhdr}
\usepackage{mathpartir}
\usepackage{xcolor}
\usepackage{hyperref}
\usepackage{xspace}
\usepackage{comment}
\usepackage{url} % for url in bib entries

%%%%%%%%%%%%%%%%%%%%%%%%%%%%%%%%%%%%%%%%
% Acronyms
%%%%%%%%%%%%%%%%%%%%%%%%%%%%%%%%%%%%%%%%
\usepackage[acronym, shortcuts]{glossaries}

\newacronym{HoTT}{HoTT}{homotopy type theory}
\newacronym{IPL}{IPL}{intuitionistic propositional logic}
\newacronym{TT}{TT}{intuitionistic type theory}
\newacronym{LEM}{LEM}{law of the excluded middle}
\newacronym{ITT}{ITT}{intensional type theory}
\newacronym{ETT}{ETT}{extensional type theory}
\newacronym{NNO}{NNO}{natural numbers object}

%%%%%%%%%%%%%%%%%%%%%%%%%%%%%%%%%%%%%%%%
% Fancy page style
%%%%%%%%%%%%%%%%%%%%%%%%%%%%%%%%%%%%%%%%
\pagestyle{fancy}
\newcommand{\metadata}[2]{
  \lhead{}
  \chead{}
  \rhead{\bfseries Homotopy Type Theory}
  \lfoot{#1}
  \cfoot{#2}
  \rfoot{\thepage}
}
\renewcommand{\headrulewidth}{0.4pt}
\renewcommand{\footrulewidth}{0.4pt}


\newcommand*{\vocab}[1]{\emph{#1}}
\newcommand*{\latin}[1]{\textit{#1}}

%%%%%%%%%%%%%%%%%%%%%%%%%%%%%%%%%%%%%%%%
% Customize list enviroonments
%%%%%%%%%%%%%%%%%%%%%%%%%%%%%%%%%%%%%%%%
% package to customize three basic list environments: enumerate, itemize and description.
\usepackage{enumitem}
\setitemize{noitemsep, topsep=0pt, leftmargin=*}
\setenumerate{noitemsep, topsep=0pt, leftmargin=*}
\setdescription{noitemsep, topsep=0pt, leftmargin=*}

%%%%%%%%%%%%%%%%%%%%%%%%%%%%%%%%%%%%%%%%
% Some really basic macros.
% (Lots of them were stolen from HoTT/Book.)
% See macros.tex in HoTT/book.
%%%%%%%%%%%%%%%%%%%%%%%%%%%%%%%%%%%%%%%%
\newcommand*{\ctx}{\Gamma}
\newcommand{\entails}{\vdash}


\newcommand*{\judgmentfont}[1]{{\normalfont\sffamily #1}}
\newcommand*{\postfixjudgment}[1]{%
  \relax\ifnum\lastnodetype>0\mskip\medmuskip\fi
  \text{\judgmentfont{#1}}%
}
\newcommand*{\prop}{\postfixjudgment{prop}}
\newcommand*{\true}{\postfixjudgment{true}}
\newcommand*{\type}{\postfixjudgment{type}}
\newcommand*{\context}{\postfixjudgment{ctx}}


\newcommand*{\truth}{\top}
\newcommand*{\conj}{\wedge}
\newcommand*{\disj}{\vee}
\newcommand*{\falsehood}{\bot}
\newcommand*{\imp}{\supset}


%%% Judgmental equality
\newcommand{\jdeq}{\equiv}
%%% Definition
\newcommand{\defeq}{\vcentcolon\equiv}
%%% Binary sums
\newcommand{\inlsym}{{\mathsf{inl}}}
\newcommand{\inrsym}{{\mathsf{inr}}}
\newcommand{\inl}{\ensuremath\inlsym\xspace}
\newcommand{\inr}{\ensuremath\inrsym\xspace}
%%% Booleans
\newcommand{\ttsym}{{\mathsf{tt}}}
\newcommand{\ffsym}{{\mathsf{ff}}}
%\newcommand{\ttrue}{\ensuremath\ttsym\xspace}
%\newcommand{\ffalse}{\ensuremath\ffsym\xspace}
%%% Pairs
\newcommand{\pair}{\ensuremath{\mathsf{pair}}\xspace}
\newcommand{\tuple}[2]{(#1,#2)}
\newcommand{\proj}[1]{\ensuremath{\mathsf{pr}_{#1}}\xspace}
\newcommand{\fst}{\ensuremath{\proj1}\xspace}
\newcommand{\snd}{\ensuremath{\proj2}\xspace}
%%% Path concatenation
\newcommand{\concat}{%
  \mathchoice{\mathbin{\raisebox{0.5ex}{$\displaystyle\centerdot$}}}%
  {\mathbin{\raisebox{0.5ex}{$\centerdot$}}}%
  {\mathbin{\raisebox{0.25ex}{$\scriptstyle\,\centerdot\,$}}}%
  {\mathbin{\raisebox{0.1ex}{$\scriptscriptstyle\,\centerdot\,$}}}
}
%%% Transport (covariant)
\newcommand{\trans}[2]{\ensuremath{{#1}_{*}\mathopen{}\left({#2}\right)\mathclose{}}\xspace}
% Natural numbers objects
\newcommand{\Nat}{\mathsf{Nat}}
\newcommand{\rec}{\ensuremath{\mathsf{rec}}\xspace}
% Sequence
\newcommand{\Seq}{\ensuremath{\mathsf{Seq}}\xspace}
% Identity type
\newcommand{\Id}[1]{\ensuremath{\mathsf{Id}_{#1}}\xspace}
% Reflection
\newcommand{\refl}[1]{\ensuremath{\mathsf{refl}_{#1}}\xspace}

% fst,snd,case,id
\renewcommand*{\fst}{\textsf{fst}}
\renewcommand*{\snd}{\textsf{snd}}
\DeclareMathOperator{\case}{\textsf{case}}
\DeclareMathOperator{\caseif}{\textsf{if}}
\DeclareMathOperator{\casesplit}{\textsf{split}}
\DeclareMathOperator{\ttrue}{\textsf{tt}\xspace}
\DeclareMathOperator{\ffalse}{\textsf{ff}\xspace}
\newcommand*{\id}{\textsf{id}}


\usepackage{proof-dashed}
\usepackage{tikz-cd}
\usepackage{amsmath}
\usepackage{lmodern}
\usepackage{microtype}

\metadata{Lee and Sojakova}{2013/10/28 and 2013/10/30}

\newtheorem{thm}{Theorem}
\newtheorem{eg}{Example}
\newcommand{\ap}{\mathsf{ap}}
\newcommand{\apd}{\mathsf{apd}}
\newcommand{\tr}{\mathsf{tr}}
\newcommand{\iseq}{\mathsf{isequiv}}
\newcommand{\qinv}{\mathsf{qinv}}
\newcommand{\happ}{\mathsf{happly}}
\newcommand{\abort}{\mathsf{abort}}
\newcommand{\seq}{\mathsf{seq}}
\newcommand{\lp}{\mathsf{loop}}
\newcommand*{\comp}{\mathbin{\circ}}
\newcommand*{\U}{\mathcal{U}}


\newtheorem*{remark}{Remark}
\newtheorem*{proposition}{Proposition}
\newtheorem*{exercise}{Exercise}
\newtheorem*{lemma}{Lemma}
\newtheorem*{definition}{Definition}


\begin{document}
\title{15-819 Homotopy Type Theory Lecture Notes}
\author{Joseph Lee and Kristina Sojakova}
\date{October 28 and 30, 2013}

\maketitle

\section{The Path Spaces of Coproducts}\label{}

Recall from last week that we intend to characterize paths in coproducts by showing
\[ \prod_{x:A+B}\prod_{x':A+B} \Id{A+B}(x,x') \simeq F(x,x') \]
where $F: (A+B) \to (A+B) \to \mathcal{U}$ is defined by nested case-analysis so that
\begin{align*}
 F(\inl(a),\inl(a'))\equiv &\ \Id{A}(a,a') \\
 F(\inr(a),\inr(a'))\equiv &\ \Id{B}(b,b') \\
 F(\inl(a),\inr(b)) \equiv &\ 0 \\
 F(\inr(a),\inl(b)) \equiv &\ 0
\end{align*}
To this end we define a function $f : \prod_{x:A+B}\prod_{x':A+B} \Id{A+B}(x,x') \to F(x,x')$ by
\[
f := \lambda x.\lambda x'.\lambda p.\; \J[F](p;z.\case(z;a.\refl{A}(a);b.\refl{B}(b)))
\]
Next we need to define a function $g:\prod_{x:A+B}\prod_{x':A+B} F(x,x')\to\Id{A+B}(x,x')$ such that $g(x,x')$ is a quasi-inverse of $f(x,x')$.
We put
{\small
\begin{align*}
g := \lambda x.\lambda x'.\case(x;&a.\case(x';a'.\lambda z:F(\inl(a),\inl(a')).\ap_{\inl}(z);b'.\lambda z:F(\inl(a),\inr(b')).\abort(z));\\
                                  &b.\case(x';a'.\lambda z:F(\inr(b),\inl(a')).\abort(z);b'.\lambda z:F(\inr(b),\inr(b')).\ap_{\inr}(z)))
\end{align*}}
We then have to exhibit terms
\begin{align*}
\alpha&:\prod_{x:A+B}\prod_{x':A+B}\prod_{u:F(x,x')}f(g(u)) =_{F(x,x')} u\\
\beta&:\prod_{x:A+B}\prod_{x':A+B}\prod_{v:\Id{A+B}(x,x')}g(f(v)) =_{\Id{A+B}(x,x')} v
\end{align*}
These terms are left as homework exercises.

\begin{exercise}
Characterize the path space of the empty type $\mathbf{0}$.
\end{exercise}

\section{The Path Spaces of Identity Types}\label{}

For a given type $A$, we would like to characterize the types $\Id{A}(\_,\_)$,
$\Id{\Id{A}(\_,\_)}(\_,\_)$, $\Id{\Id{\Id{A}(\_,\_)}(\_,\_)}(\_,\_)$, and so on. While this is possible for certain specific types such as $\mathbf{0}, \mathbf{1}, A \times B, A \to B, A + B$ and $\Nat$, it can be very difficult for other types, even seemingly simple ones. For example, determining the loop-spaces of $n$-spheres - i.e., path spaces based at a single point, denoted by $\Omega(S^n)$ - is a famous open problem in algebraic topology.

What we can say, however, is that if two types are equivalent then so are their path spaces:

\begin{lemma} If $f : A \to B$ is an equivalence, then so is \
$\ap_f:\Id{A}(a,a')\to\Id{B}(f(a),f(a'))$.
\end{lemma}

\begin{proof}
Because $f$ is an equivalence, it has a quasi-inverse $f^{-1}:B\to A$ and we have the following coherences:

\begin{itemize}
\item $\alpha:\prod_{a:A} f^{-1}(f(a)) =_A a$
\item $\beta:\prod_{b:B} f(f^{-1}(b) =_B b$
\end{itemize}

In order to show that $\ap_f$ is an equivalence, it suffices to give a
quasi-inverse $\ap_f^{-1}:\Id{B}(f(a),f(a'))\to\Id{A}(a,a')$, which we define by
\[
\ap_f^{-1}(q) := \alpha(a)^{-1} \cdot \ap_{f^{-1}}(q) \cdot \alpha(a')
\]
We now need to construct coherences
\begin{align*}
\gamma&:\prod_{p:\Id{A}(a,a')}\ap_f^{-1}(\ap_f(p)) =_{\Id{A}(a,a')} p \\
\delta&:\prod_{q:\Id{B}(f(a),f(a'))}\ap_f(\ap_f^{-1}(q)) =_{\Id{B}(f(a),f(a'))} q
\end{align*}
This will imply that $\ap_f^{-1}$ is indeed a quasi-inverse of $\ap_f$ and thus both are equivalences We leave these as exercises.
\end{proof}

\begin{exercise} Define the coherence $\gamma$ in the above proof by path induction.
\end{exercise}

\begin{exercise}
Define the coherence $\delta$ in the above proof as follows:
\begin{enumerate}
\item Use the naturality of $\beta$ to show that
$$\beta(f(a))^{-1} \cdot \ap_f(\ap_{f^{-1}}(q)) \cdot \beta(f(a')) =_{\Id{B}(f(a),f(a'))} q$$
\item Use the naturality of $\alpha$ to show that
$$\alpha(f^{-1}(f(a)))^{-1} \cdot \ap_{f^{-1}}(\ap_f(\ap_{f^{-1}}(q))) \cdot \alpha(f^{-1}(f(a'))) =_{\Id{A}(f^{-1}(f(a)),f^{-1}(f(a')))} \ap_{f^{-1}}(q)$$
\item Use the naturality of $\alpha$ to show that
\begin{align*}
& \alpha(f^{-1}(f(a))) =_{\Id{A}((f^{-1}(f(f^{-1}(f(a))))), f^{-1}(f(a)))}  \ap_{f^{-1}}(\ap_f(\alpha(a)))
\end{align*}
and similarly for $a'$.
\item Use 1), 2), 3) and the naturality of $\beta$ to obtain the desired conclusion
$$\ap_f(\alpha(a)^{-1} \cdot \ap_{f^{-1}}(q) \cdot \alpha(a')) =_{\Id{B}(f(a),f(a'))} q$$
\end{enumerate}
\end{exercise}

\section{Transport Properties of Identity}\label{}

The identity type family on a type $A$ can be thought of as a function $Id{A}(\_,\_) : A \to A \to \mathcal{U}$. Keeping the first argument $x : A$ fixed yields a type family
$Id{A}(x,\_) : A \to \U$. For any path $q : y =_A y'$, the fibers $\Id{A}(x,y)$ and $\Id{A}(x, y')$ are related by the transport function
\[ \tr[z.\Id{A}(x, z)](q) : \Id{A}(x, y) \to \Id{A}(x, y') \]
Can we give an explicit description of this function? Clearly we can construct a function of the desired type "manually" by taking a $p : x =A y$ and concatenating it
with $q$ to obtain $p \cdot q : x =_A y'$. Fortunately, it turns out this is precisely characterizes the behavior of the transport function, up to a propositional equality:
\begin{lemma}
For any term $x : A$, and paths $q : y =_A y'$, $p : x =_A y$ we have
\[ \tr[z.\Id{A}(x, z)](q)(p) =_{\Id{A}(x,y')} p \cdot q \]
\end{lemma}
We have a similar characterization of transport in the case when the second
argument is fixed:
\begin{lemma}
For any term $y : A$, and paths $q : x =_A x'$, $p : x =_A y$ we have
\[ \tr[z.\Id{A}(z, y)](q)(p) =_{\Id{A}(x',y)} q^{- 1} \cdot p \]
\end{lemma}
We notice that in the first case, we use the path $q$ as-is whereas in the second
case we first have to invert it. This can be described in category-theoretic terms as
saying that the type family $\Id{A}$ is \emph{covariant in the second argument and contravariant
in the first.}

Finally, we can consider the case of loops when the base point is allowed to vary:
\begin{lemma}
For any paths $q : x =_A y$, $p : x =_A x$ we have
\[ \tr[z.\Id{A}(z, z)](q)(p) =_{\Id{A}(y,y)} q^{ -1} \cdot p \cdot q \]
\end{lemma}
All these lemmas follow by straightforward path induction.




\section{Justifying the Identity-Elimination Rule}\label{}

Recall the identity elimination rule:
{\small \[\infer[\Id{}\text{-E}]{\Gamma \entails \J[x.y.z.C](P,x.Q): [M,N,P/x,y,z]C}
    {\Gamma \entails P:\Id{A}(M,N) & \Gamma,x{:}A,y{:}A,z{:}\Id{A}(x,y) \entails C \type & \Gamma, x{:}A\entails Q:[x,x,\refl{A}(x)/x,y,z]C}\]}
In Extensional Type Theory (ETT), this rule is a consequence of the identity reflection and UIP rules and thus has no special status.

In Intensional Type Theory (ITT), this rule can be understood as an induction principle: since the only way we can construct a proof of equality $P : M =_A N$ is by reflexivity in the case when
$M \equiv N$, in order to prove $C[M,N,P]$ it is sufficient to prove $C[x,x,\refl{A}(x)]$ for an arbitrary $x : A$. This intuition is justified by the following very important (and highly nontrivial) theorem:
\begin{thm}
In an empty context, two terms are propositionally equal if and only if they are definitionally equal and any identity proof is necessarily a reflexivity. In other words, if $\vdash P:\Id{A}(M,N)$, then $\vdash M\equiv N: A$ and $\vdash P\equiv\refl{A}(M,M):\Id{A}(M,N)$.
\end{thm}

In HoTT, it is no longer the case that every proof of equality is a reflexivity. For example, we have the following non-trivial identity proofs:

\begin{itemize}
\item $\mathsf{funext}(H): f =_{A\to B} g$, where $H : \prod_{a:A} f(a) =_B g(a)$
\item $\mathsf{ua}(E): A =_{\mathcal{U}} B$, where $E : \sum_{f:A\to B}\iseq(f)$
\item $\seq: 0 =_{\mathbb{I}} 1$, where $\mathbb{I}$ is the interval type
\item $\lp: b =_{S^1} b$, where $S^1$ is the circle type
\end{itemize}

This suggests that in HoTT, terms of an identity type should not be thought of purely as proofs of identity but rather as paths between terms. Since there can be potentially many distinct paths between two terms, the identity elimination rule should no longer be thought of as an induction principle.

The presence of nontrivial paths, however, poses a serious problem with the computational interpretation of HoTT: for example, what should $\J[\_](\mathsf{funext}(H);x.Q)$, $\J[\_](\mathsf{ua}(E))$, or $\J[\_](\seq)$ compute to?

Even leaving aside univalence and higher inductive types, the addition of the function extensionality axiom to ITT poses a problem with computation. In ETT, we get function extensionality for free from the identity reflection rule. In Observational Type Theory (OTT), which combines intensional and extensional aspects, we get function extensionality with some special arrangements. Both ETT and OTT admit a computational interpretation, albeit by different means.

The computational interpretation of HoTT is currently a principal open problem. So how do we justify the $\J$-rule as a suitable identity elimination rule? Given

\begin{itemize}
\item $C : \prod_{x : A} \prod{y : A}, \Id{A}{x}{y} \to \mathcal{U}$
\item $M,N:A$ and $P:M=_A N$
\item $x:A\vdash Q:C[x,x,\refl{A}(x)]$
\end{itemize}
why should there exist a term $\J[x.y.z.C](P;x.Q):C[M,N,P]$?

Plugging $M$ into $Q$, we obtain a term $Q[M]:C[M,M,\refl{A}(M)]$. Similarly, plugging $M$ into $C$ yields a type family $C[M] : \prod_{y:A}(p:M=_A y)\to\mathcal{U}$, which can be equivalently understood as the function \[\lambda z. C[M, \pi_1(z), \pi_2(z)] : \big(\sum_{y:A} \Id{A}(M,y)\big)\to\mathcal{U} \]

Since functions are supposed to be functorial, constructing a path $\gamma$ from $(M,\refl{A}(M))$ to $(N,P)$ in the type $\sum_{y:A} \Id{A}(M,y)$ would give us a term
\[ \ap_{\lambda z. C[M, \pi_1(z), \pi_2(z)]}(\gamma) : C[M,M,\refl{A}(M)] =_\mathcal{U} C[M,N,P] \]
We could thus obtain our desired conclusion of the $\J$-rule as \[ \tr[x : \mathcal{U}.x](\ap_{\lambda z. C[M, \pi_1(z), \pi_2(z)]}(\gamma))(Q[M]) : C[M,N,P]\]
In order to construct a path {\small $\gamma : (M,\refl{A}(M)) =_{\sum_{y:A} \Id{A}(M,y)} (N,P)$}, we need a characterization of path spaces of $\Sigma$-types, outlined in the following exercise:

\begin{exercise} Characterize the path space of the type $\Sigma_{x:A} B$ by constructing a term of type
\small {\[ \prod_{p,p':\sum_{x:A}B(x)} \big( \Id{\sum_{x:A}B(x)}(p,p')\simeq\sum_{q:\pi_1(p)=_A\pi_1(p')}\pi_2(p) =^{x.B}_q \pi_2(p') \big) \]}
\end{exercise}

The above exercise tells us that in order to construct a path from $(M,\refl{A}(M))$ to $(N,P)$ in the type $\sum_{y:A}M=_A y$, it is sufficient to construct an element of the type
$\sum_{q:M=_A N}\refl{A}(M) =^{y.M=_A y}_q P$. The natural choice for the first component of the pair is the path $P : M=_A N$ itself. It thus remains to show that $\refl{A}(M) =^{y.M=_A y}_P P$. In particular, this means showing that \[ \tr[y.M=_A y](P)(\refl{A}(M)) =_{\Id{A}(M,N)} P \]
By the first lemma in Section.~3, we have $\tr[y.M=_A y](P)(\refl{A}(M)) =_{\Id{A}(M,N)} \refl{A}(M) \cdot P$. Since the left-hand evaluates to $P$, we are done.

\section{Introduction to Homotopy Types}
One way to see HoTT is that Homotopy \emph{Type Theory} is \emph{Homotopy Type} Theory.
That is, HoTT can be thought of as the theory of homotopy types. We have already
encountered several examples of the homotopy type \emph{set}, also sometimes called an
\emph{h-set} or a \emph{0-type}; however, we have not explicitly labeled them as such. We have the
following definition:

\begin{definition} A type $A$ is called a \emph{set} (or a \emph{0-type}), if for all $x, y : A$ and $p, q : x =_A y$,
we have that $p =_{\Id{A}(x,y)} q$. In other words, the following type is inhabited:
  \[ \mathsf{isSet}(A) \defeq \prod_{x,y:A} \prod_{p,q:\Id{A}(x,y)} p =_{\Id{A}(x,y)} q \]
\end{definition}
Intuitively, the type $A$ can be viewed as "discrete up to homotopy". The familiar example of the type $\Nat$ of natural numbers (unsurisingly) turns out to be a set.
More about this and other Homotopy Types next week!

\end{document}
