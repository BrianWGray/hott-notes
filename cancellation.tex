\documentclass{article}
\usepackage{latexsym}
\usepackage{amsmath}
\usepackage{amssymb}
\usepackage{amsthm}
\usepackage{mathpazo}
\newcommand{\zeroty}{\mathsf{void}}
\newcommand{\abort}[1]{\mathsf{abort}(#1)}
\newcommand{\sumty}[2]{{#1}+{#2}}
\newcommand{\inleft}[1]{\mathsf{inl}(#1)}
\newcommand{\inright}[1]{\mathsf{inr}{(#1)}}
\newcommand{\sumcase}[5]{\mathsf{case}({#1};{#2}{.}{#3};{#4}{.}{#5})}
\newcommand{\idty}[3]{{#2}\mathbin{=_{#1}}{#3}}
\newcommand{\refl}[2]{\mathsf{refl}_{#1}({#2})}
\newcommand{\idelim}[4]{\mathsf{J}[{#1}]({#2};{#3}{.}{#4})}
\newcommand{\family}[2]{{#1}.{#2}}
\newcommand{\familytwo}[3]{\family{#1}{\family{#2}{#3}}}
\newcommand{\familythree}[4]{\family{#1}{\familytwo{#2}{#3}{#4}}}
\newcommand{\prodty}[3]{\prod_{{#1}{:}{#2}}{#3}}
\newcommand{\sigty}[3]{\sum_{{#1}{:}{#2}}{#3}}
\newcommand{\lam}[2]{\lambda {#1}{.}{#2}}
\newcommand{\app}[2]{{#1}({#2})}
\newcommand{\pa}[1]{\mathsf{ap}_{#1}}
\newcommand{\univty}{\mathcal{U}}

\newcommand{\iv}[1]{#1^{-1}}
\newcommand{\concat}[2]{{#1}\cdot{#2}}

\newcommand{\transport}[2]{\mathsf{trans}[#1](#2)}

\newtheorem{theorem}{Theorem}
\newtheorem{defn}[theorem]{Definition}
\newtheorem{lemma}[theorem]{Lemma}
\newtheorem{corollary}[theorem]{Corollary}

\title{The Cancellation Method}
\date{Fall, 2013}

\begin{document}

\section{Co-product Paths}

As in the HoTT book, we wish to prove the following characterization of paths in a coproduct type $A+B$:
\begin{enumerate}
\item $\idty{\sumty{A}{B}}{\inleft{M}}{\inleft{M'}} \simeq \idty{A}{M}{M'}$;
\item $\idty{\sumty{A}{B}}{\inright{N}}{\inright{N'}} \simeq \idty{B}{N}{N'}$;
\item $\idty{\sumty{A}{B}}{\inleft{\_}}{\inright{\_}} \simeq \zeroty$;
\item $\idty{\sumty{A}{B}}{\inright{\_}}{\inleft{\_}} \simeq \zeroty$.
\end{enumerate}

Define the family $$u:\sumty{A}{B},v:\sumty{A}{B}\vdash F[u,v]:\univty$$ so that the following definitional equivalences hold:
\begin{enumerate}
\item $F[\inleft{M},\inleft{M'}]\equiv \idty{A}{M}{M'}$;
\item $F[\inright{M},\inright{M'}]\equiv \idty{A}{M}{M'}$;
\item $F[\inleft{\_},\inright{\_}]\equiv\zeroty$;
\item $F[\inright{\_},\inleft{\_}]\equiv\zeroty$.
\end{enumerate}
This is easily achieved by a nested case analysis on $u$ and $v$, respectively, with motive $\univty$ in each case.  The correspondence between the defnition of $F$ and the desired theorem is evident.  The point is that $F$ cancels the injections definitionally, so that no particular mention need be made of this in the proof.

\begin{lemma}
  \label{L-lemma}
  There is a term $L$ of type $\prodty{z}{\sumty{A}{B}}{F[z,z]}$ such that $\app{L}{\inleft{x}}\equiv \refl{A}{x}$ and $\app{L}{\inright{y}}\equiv \refl{B}{y}$.
\end{lemma}
\begin{proof}
  Construct $L$ by abstracting over $z$, then performing a case analysis on $z$:
  \begin{enumerate}
  \item $x:A\vdash \refl{A}{x}:F[\inleft{x},\inleft{x}]$.
  \item $y:B\vdash \refl{A}{y}:F[\inright{y},\inright{y}]$.
  \end{enumerate}
\end{proof}

We wish to show that $$\prodty{z,z'}{\sumty{A}{B}}{\idty{\sumty{A}{B}}{z}{z'}}\simeq F[z,z'],$$ from which the desired result follows immediately by definition of $F$.

First we exhibit the function
$$f:\prodty{z}{\sumty{A}{B}}{\prodty{z'}{\sumty{A}{B}}{\idty{\sumty{A}{B}}{z}{z'}\to F[z,z']}}$$
given as follows:
$$\lam{z}{\lam{z'}{\lam{p}{\idelim{\familythree{u}{v}{\_}{F[u,v]}}{p}{x}{\app{L}{x}}}}}.$$
Lemma~\ref{L-lemma} does all the work: if $M:\sumty{A}{B}$, then $$\app{\app{\app{f}{M}}{M}}{\refl{\sumty{A}{B}}{M}}\equiv \app{L}{M}.$$

We then exhibit a quasi-inverse for $f$,
$$g:\prodty{z}{\sumty{A}{B}}{\prodty{z'}{\sumty{A}{B}}{F[z,z']\to\idty{\sumty{A}{B}}{z}{z'}}}$$
given by a nested case analysis on $z$ and $z'$ based on the following data:
\begin{enumerate}
\item $x:A,x':A,z:F[\inleft{x},\inleft{x'}]\vdash \app{\pa{\inleft{-}}}{z}:\idty{\sumty{A}{B}}{\inleft{x}}{\inleft{x'}}$;
\item $x:A,y':B,z:F[\inleft{x},\inright{y'}]\vdash \abort{z}:\idty{\sumty{A}{B}}{\inleft{x}}{\inright{y'}}$.
\item $y:B,y':B,z:F[\inright{x},\inright{x'}]\vdash \app{\pa{\inright{-}}}{z}:\idty{\sumty{A}{B}}{\inright{x}}{\inright{x'}}$;
\item $x:A,y':B,z:F[\inright{x},\inleft{y'}]\vdash \abort{z}:\idty{\sumty{A}{B}}{\inright{x}}{\inleft{y'}}$.
\end{enumerate}
Here we are relying on the definitional properties of the family $F$ to justify the given typings.  Notice that the $\lambda$-abstraction of the third argument to $g$ must occur \emph{inside} the case analysis in order to propagate the correct branch information (\textit{cf}. the hacky treatment of this issue in Haskell's so-called GADT's.)

\begin{lemma} 
  \label{g-lemma}
  The following types are inhabited:
  \begin{enumerate}
  \item $x:A\vdash \_ : \idty{\sumty{A}{B}}{\app{\app{\app{g}{\inleft{x}}}{\inleft{x}}}{\refl{A}{x}}}{\refl{\sumty{A}{B}}{\inleft{x}}}$;
  \item $y:B\vdash \_ : \idty{\sumty{A}{B}}{\app{\app{\app{g}{\inright{x}}}{\inright{x}}}{\refl{B}{y}}}{\refl{\sumty{A}{B}}{\inright{y}}}$.
  \end{enumerate}
\end{lemma}

To complete the proof we need only exhibit witnesses to the fact that for $z,z':\sumty{A}{B}$, the function  $g'=\app{\app{g}{z}}{z'}$ is right and left inverse to the function $f'=\app{\app{f}{z}}{z'}$, up to higher homotopy.
\begin{enumerate}
\item $z:\sumty{A}{B},z':\sumty{A}{B},w:F(z,z')\vdash \alpha : \idty{F[z,z']}{\app{f'}{\app{g'}{w}}}{w}$.
\item $z:\sumty{A}{B},z':\sumty{A}{B},w:\idty{\sumty{A}{B}}{z}{z'}\vdash \beta : \idty{\idty{\sumty{A}{B}}{z}{z'}}{\app{g'}{\app{f'}{w}}}{w}$.
\end{enumerate}

The first is proved by a nested case analysis on $z$ and $z'$, using the definitional equivalences governing $F$, either aborting, or performing a path induction on $w$, appealing to Lemma~\ref{g-lemma} to complete the proof.  The second is proved by induction on $w$, using a case analysis on the generic $u:\sumty{A}{B}$ so that we may appeal to Lemma~\ref{g-lemma} to complete the proof.

\section{Application to Paths}

The idea is to adapt the above proof for coproducts, except using propositional equivalences for the family $F$ obtained from the assumption that $f$ has a quasi-inverse.

We are given $f:A\to B$ and the following data showing that $f$ has a quasi-inverse:
\begin{enumerate}
\item $\iv{f}:B\to A$;
\item $\alpha:\prodty{a}{A}{\idty{A}{\app{\iv{f}}{\app{f}{a}}}{a}}$;
\item $\beta:\prodty{b}{B}{\idty{B}{\app{f}{\app{\iv{f}}{b}}}{b}}$.
\end{enumerate}

We are to show that $\pa{f}$ has a quasi-inverse, which amounts to giving the following data:
\begin{enumerate}
\item $\iv{\pa{f}}$, which is taken to be $\lam{q}{\concat{\concat{\iv{\app{\alpha}{a}}}{\app{\pa{\iv{f}}}{q}}}{\app{\alpha}{a'}}}$;
\item $\alpha':\prodty{a}{A}{\prodty{a'}{A}{\prodty{p}{\idty{A}{a}{a'}}{\idty{\idty{A}{a}{a'}}{\app{\iv{\pa{f}}}{\app{\pa{f}}{p}}}{p}}}}$;
\item $\beta':\prodty{a}{A}{\prodty{a'}{A}{\prodty{q}{\idty{B}{\app{f}{a}}{\app{f}{a'}}}{\idty{\idty{B}{\app{f}{a}}{\app{f}{a'}}}{\app{\pa{f}}{\app{\iv{\pa{f}}}{q}}}{q}}}}$.
\end{enumerate}

The construction of $\alpha'$ takes the form of a path induction on $p$,
reducing the problem to the case of reflexivity for a generic $x$ of type $A$.
This presents no difficulties, because the end points of the path in question
are variables that also occur in the motive.

The construction of $\beta'$ is more difficult, because the evident source of
path induction, $q$, has as end points $\app{f}{a}$ and $\app{f}{a'}$, which
will appear in the conclusion of the proof.  More precisely, if $F$ is the
motive for a path induction on $q$, then the conclusions will be of the form
$F[\app{f}{a},\app{f}{a'},q]$, whereas the desired conclusion involves just $a$,
$a'$, and $q$.

We must choose the path on which to induct and the motive for the induction in
such a way that the desired conclusion follows from the corresponding instance
of the motive.  One approach is to induct on $\app{\pa{\iv{f}}}{q}$, which
has type
$$\idty{A}{\app{\iv{f}}{\app{f}{a}}}{\app{\iv{f}}{\app{f}{a'}}}.$$
Using the quasi-inverse for $f$ this type may be shown to be propositionally
equal to the type $\idty{A}{a}{a'}$, so any element of the former type may be
transported to the latter, and \textit{vice versa}.

An appropriate motive for the induction is the type family
$u{:}A,v{:}A,w{:}\idty{A}{u}{v}\vdash F:\univty$ defined by the equality type
\begin{displaymath}
  \idty{\idty{B}{\app{f}{u}}{\app{f}{v}}}{\app{\pa{f}}{\concat{\concat{\iv{\app{\alpha}{u}}}{\app{\pa{\iv{f}}}{\app{\pa{f}}{w}}}}{\app{\alpha}{v}}}}{\pa{f}{w}}.
\end{displaymath}
With $F$ as motive the path induction on $\app{\pa{\iv{f}}}{q}$ yields an inhabitant
of the type
$${F[\app{\iv{f}}{\app{f}{a}},\app{\iv{f}}{\app{f}{a'}},\app{\pa{\iv{f}}}{q}]},$$
which is equal to
\begin{displaymath}
  \idty{}
  {\app{\pa{f}}
    {\concat
      {\concat
        {\iv{\app{\alpha}{\app{\iv{f}}{\app{f}{a}}}}}
        {\app{\pa{\iv{f}}}{\app{\pa{f}}{\app{\pa{\iv{f}}}{q}}}}
      }
      {\app{\alpha}{\app{\iv{f}}{\app{f}{a'}}}}
    }
  }
  {\app{\pa{f}}{\app{\pa{\iv{f}}}{q}}}.
\end{displaymath}
Using the quasi-inverse for $f$, we may show that this equation is equal to the equation
\begin{displaymath}
  \idty{}
  {\app{\pa{f}}
      {\concat
        {\concat
          {\iv{\app{\alpha}{{a}}}}
          {\app{\pa{\iv{f}}}{q}}
        }
        {\app{\alpha}{a'}}
      }
    }
    {q},
\end{displaymath}
which is the desired conclusion.

It remains to show, then, that 
\begin{displaymath}
  x{:}A \vdash \_ : F[x,x,\refl{A}{x}],
\end{displaymath}
which is to say that 
\begin{displaymath}
  x{:}A \vdash \_ : \idty{}{\app{\pa{f}}{\concat{\concat{\iv{\app{\alpha}{x}}}{\app{\pa{\iv{f}}}{\app{\pa{f}}{\refl{A}{x}}}}}{\app{\alpha}{x}}}}{\app{\pa{f}}{\refl{A}{x}}}.
\end{displaymath}
Now $\app{\pa{f}}{\refl{A}{x}}\equiv\refl{B}{\app{f}{x}}$, and
$\app{\iv{\pa{f}}}{\refl{B}{\app{f}{x}}}$, so this amounts to
showing
\begin{displaymath}
  x{:}A \vdash \_ : \idty{}{\app{\pa{f}}{\concat{\concat{\iv{\app{\alpha}{x}}}{\refl{B}{\app{\iv{f}}{\app{f}{x}}}}}{\app{\alpha}{x}}}}{\refl{B}{\app{f}{x}}}.
\end{displaymath}
Using the unit laws for path concatenation this reduces to showing
\begin{displaymath}
  y:B \vdash \_ : \idty{B}{\app{\pa{f}}{\concat{\iv{\app{\alpha}{y}}}{\app{\alpha}{y}}}}{\refl{B}{y}}.
\end{displaymath}
Applying the inverse law for paths, this is just
\begin{displaymath}
  y:B \vdash \_ : \idty{B}{\app{\pa{f}}{\refl{B}{\app{\iv{f}}{y}}}}{\refl{B}{y}}.
\end{displaymath}
Finally, $\app{\pa{f}}{\refl{B}{\app{\iv{f}}{y}}}\equiv \refl{B}{\app{f}{\app{\iv{f}}{y}}}$, and we have
\begin{displaymath}
  y:B \vdash \_ : \idty{B}{\refl{B}{\app{f}{\app{\iv{f}}{y}}}}{\refl{B}{y}},
\end{displaymath}
using the quasi-inverse of $f$, which completes the argument.

\smallskip

Throughout I am tacitly using the principle that $$\transport{\family{x}{x}}{p}:A\to A'$$ whenever $p:\idty{\univty}{A}{A'}$, which is to say that one may transport objects of a type $A$ to an object of an equal type $A'$ in the same universe.  This move corresponds to the implicit uses of definitional equality of types in the characterization of the paths in a coproduct type.

\end{document}
