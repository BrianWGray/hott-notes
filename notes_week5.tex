\documentclass[12pt]{article}
\usepackage[letterpaper]{geometry}                                              
                                                                                
%%% This file can never be completed.
%%% If you need something but cannot find it,
%%% contact the TA Favonia!

%%%%%%%%%%%%%%%%%%%%%%%%%%%%%%%%%%%%%%%%
% Basic packages
%%%%%%%%%%%%%%%%%%%%%%%%%%%%%%%%%%%%%%%%
\usepackage{amsmath,amsthm,amssymb}
\usepackage{fancyhdr}
\usepackage{mathpartir}
\usepackage{xcolor}
\usepackage{hyperref}
\usepackage{xspace}
\usepackage{comment}
\usepackage{url} % for url in bib entries

%%%%%%%%%%%%%%%%%%%%%%%%%%%%%%%%%%%%%%%%
% Acronyms
%%%%%%%%%%%%%%%%%%%%%%%%%%%%%%%%%%%%%%%%
\usepackage[acronym, shortcuts]{glossaries}

\newacronym{HoTT}{HoTT}{homotopy type theory}
\newacronym{IPL}{IPL}{intuitionistic propositional logic}
\newacronym{TT}{TT}{intuitionistic type theory}
\newacronym{LEM}{LEM}{law of the excluded middle}
\newacronym{ITT}{ITT}{intensional type theory}
\newacronym{ETT}{ETT}{extensional type theory}
\newacronym{NNO}{NNO}{natural numbers object}

%%%%%%%%%%%%%%%%%%%%%%%%%%%%%%%%%%%%%%%%
% Fancy page style
%%%%%%%%%%%%%%%%%%%%%%%%%%%%%%%%%%%%%%%%
\pagestyle{fancy}
\newcommand{\metadata}[2]{
  \lhead{}
  \chead{}
  \rhead{\bfseries Homotopy Type Theory}
  \lfoot{#1}
  \cfoot{#2}
  \rfoot{\thepage}
}
\renewcommand{\headrulewidth}{0.4pt}
\renewcommand{\footrulewidth}{0.4pt}


\newcommand*{\vocab}[1]{\emph{#1}}
\newcommand*{\latin}[1]{\textit{#1}}

%%%%%%%%%%%%%%%%%%%%%%%%%%%%%%%%%%%%%%%%
% Customize list enviroonments
%%%%%%%%%%%%%%%%%%%%%%%%%%%%%%%%%%%%%%%%
% package to customize three basic list environments: enumerate, itemize and description.
\usepackage{enumitem}
\setitemize{noitemsep, topsep=0pt, leftmargin=*}
\setenumerate{noitemsep, topsep=0pt, leftmargin=*}
\setdescription{noitemsep, topsep=0pt, leftmargin=*}

%%%%%%%%%%%%%%%%%%%%%%%%%%%%%%%%%%%%%%%%
% Some really basic macros.
% (Lots of them were stolen from HoTT/Book.)
% See macros.tex in HoTT/book.
%%%%%%%%%%%%%%%%%%%%%%%%%%%%%%%%%%%%%%%%
\newcommand*{\ctx}{\Gamma}
\newcommand{\entails}{\vdash}


\newcommand*{\judgmentfont}[1]{{\normalfont\sffamily #1}}
\newcommand*{\postfixjudgment}[1]{%
  \relax\ifnum\lastnodetype>0\mskip\medmuskip\fi
  \text{\judgmentfont{#1}}%
}
\newcommand*{\prop}{\postfixjudgment{prop}}
\newcommand*{\true}{\postfixjudgment{true}}
\newcommand*{\type}{\postfixjudgment{type}}
\newcommand*{\context}{\postfixjudgment{ctx}}


\newcommand*{\truth}{\top}
\newcommand*{\conj}{\wedge}
\newcommand*{\disj}{\vee}
\newcommand*{\falsehood}{\bot}
\newcommand*{\imp}{\supset}


%%% Judgmental equality
\newcommand{\jdeq}{\equiv}
%%% Definition
\newcommand{\defeq}{\vcentcolon\equiv}
%%% Binary sums
\newcommand{\inlsym}{{\mathsf{inl}}}
\newcommand{\inrsym}{{\mathsf{inr}}}
\newcommand{\inl}{\ensuremath\inlsym\xspace}
\newcommand{\inr}{\ensuremath\inrsym\xspace}
%%% Booleans
\newcommand{\ttsym}{{\mathsf{tt}}}
\newcommand{\ffsym}{{\mathsf{ff}}}
%\newcommand{\ttrue}{\ensuremath\ttsym\xspace}
%\newcommand{\ffalse}{\ensuremath\ffsym\xspace}
%%% Pairs
\newcommand{\pair}{\ensuremath{\mathsf{pair}}\xspace}
\newcommand{\tuple}[2]{(#1,#2)}
\newcommand{\proj}[1]{\ensuremath{\mathsf{pr}_{#1}}\xspace}
\newcommand{\fst}{\ensuremath{\proj1}\xspace}
\newcommand{\snd}{\ensuremath{\proj2}\xspace}
%%% Path concatenation
\newcommand{\concat}{%
  \mathchoice{\mathbin{\raisebox{0.5ex}{$\displaystyle\centerdot$}}}%
  {\mathbin{\raisebox{0.5ex}{$\centerdot$}}}%
  {\mathbin{\raisebox{0.25ex}{$\scriptstyle\,\centerdot\,$}}}%
  {\mathbin{\raisebox{0.1ex}{$\scriptscriptstyle\,\centerdot\,$}}}
}
%%% Transport (covariant)
\newcommand{\trans}[2]{\ensuremath{{#1}_{*}\mathopen{}\left({#2}\right)\mathclose{}}\xspace}
% Natural numbers objects
\newcommand{\Nat}{\mathsf{Nat}}
\newcommand{\rec}{\ensuremath{\mathsf{rec}}\xspace}
% Sequence
\newcommand{\Seq}{\ensuremath{\mathsf{Seq}}\xspace}
% Identity type
\newcommand{\Id}[1]{\ensuremath{\mathsf{Id}_{#1}}\xspace}
% Reflection
\newcommand{\refl}[1]{\ensuremath{\mathsf{refl}_{#1}}\xspace}

% fst,snd,case,id
\renewcommand*{\fst}{\textsf{fst}}
\renewcommand*{\snd}{\textsf{snd}}
\DeclareMathOperator{\case}{\textsf{case}}
\DeclareMathOperator{\caseif}{\textsf{if}}
\DeclareMathOperator{\casesplit}{\textsf{split}}
\DeclareMathOperator{\ttrue}{\textsf{tt}\xspace}
\DeclareMathOperator{\ffalse}{\textsf{ff}\xspace}
\newcommand*{\id}{\textsf{id}}
                                                                  
                                                                                
\usepackage{proof-dashed}                                                       
\usepackage{tikz-cd}                                                            
\usepackage{amsmath}                                                            
\usepackage{lmodern}                                                            
\usepackage{microtype}                                                          

\metadata{Nathan Fulton}{2013/10/14 and 2013/10/16}

% Definitions specific to these notes.
\newcommand{\suc}{\mathsf{succ}}
\newcommand{\Ap}{\mathsf{ap}}
\newcommand{\UU}{\ensuremath{\mathcal{U}}\xspace} % From the book macros
\newcommand{\Bool}{\mathsf{Bool}}

\newtheorem*{toc}{Theorem of Choice}
\newtheorem*{foe}{Failure of Extensionality}
\newtheorem*{hedberg}{Hedberg's Theorem}
\newtheorem{thm}{Theorem}
\newtheorem{eg}{Example}
\newtheorem*{remark}{Remark}

\begin{document}
\title{15-819 Homotopy Type Theory Lecture Notes} 
\author{Nathan Fulton}
\date{October 9 and 11, 2013}

\maketitle

\section{Contents}\label{sec:contents}

These notes summarize and extend two lectures from Bob Harper's Homotopy Type Theory course.  
The cumulative hierarchy of type universes, Extensional Type theory, the $\infty$-groupoid
structure of types and iterated identity types are presented.

\section{Motivation and Overview}\label{sec:oview}

Recall from previous lectures the definitions of functionality and transport.
Functionality states that functions preserve identity; that is, domain elements
equal in their type map to equal elements in the codomain.
Transportation states the same for type families.  Traditionally, this means
that if $a =_A a'$, then $B[a] \true$ iff $B[a'] \true$.  In proof-relevant
mathematics, this logical equivalence is generalized to a statement about
identity in the family: if $a =_A a'$, then $B[a] =_B B[a']$.

Transportation can be thought of in terms of functional extensionality.
Unfortunately, extensionality fails in ITT.  One way to recover 
extensionality, which comports with traditional mathematics, is to reduce all 
identity to reflexivity.  This approach, called Extensional Type theory (ETT), 
provides a natural setting for set-level mathematics.

The HoTT perspective on ETT is that the path structure of types need not 
be limited to that of strict sets.
The richer path structure of an $\infty$-groupoid is induced by the induction
principle for identity types.
Finding a type-theoretic description of this behavior (that is, introduction,
elimination and computation rules which comport with Gentzen's Inversion
Principle) is an open problem.


\section{The Cumulative Hierarchy of Universes}\label{sec:universe}

In previous formulations of ITT, we used the judgement $A \type $ 
when forming types.
In this setting, many types are natural to write down but impossible to form.
As a running example for the section, consider the following:
$\caseif{M,\overline{17},\ttrue} : \caseif{M, \Nat, \Bool}$.
Assuming the well-formedness of the type, elimination rules behave as expected:
$\overline{17} : \caseif{\ttrue, \Nat, \Bool} \equiv \Nat$
and $\ttrue : \caseif{\ffalse, \Nat, \Bool} \equiv \Bool$.

Forming this type is not possible using the current formation rule for 
$\mathsf{\overline{if}}$.
Type universes address this shortcoming by generalizing type 
formation rules.  A recursively generated cumulative hierarchy of universes 
($\UU_i$) is introduced.  Instead of defining type formation in terms of a judgement
$A \type$, formation rules state the relative location of relevant 
types in the hierarchy; that is, judgements that $A \type$ are replaced with
judgements of the form $A : \UU_i$.

The definition of type universes includes three new rules\footnote{See A1.1 and 
A2.3 of the HOTT book for discussion.}

\begin{equation*}
  \infer[\UU\text{-intro}]{\ctx \vdash \UU_i : \UU_{i+1}}{\ctx \mathsf{ctx}}  
  \hspace{50pt}
  \infer[\UU\text{-cumul}]{\ctx \vdash A : \UU_{i+1}}{\ctx \vdash A : \UU_i}
  \hspace{50pt}
  \infer[\UU\text{-}\equiv]{A \equiv B : \UU_{i+1}}{A \equiv B : \UU_i}
\end{equation*}

The $\UU\text{-intro}$ rule introduces an unbounded hierarchy of universes, each of which
inhabits the next universe.
The second rule states that these universes are cumulative, and the third 
ensures that equality is preserved in higher universes.  The
$\UU\text{-}\equiv$ rule is a derived rule in the HoTT book presentation.

In addition to these rules, every type formation rule establishes relative
positions of relevant types.  For example\footnote{See appendix 2 of \cite{HoTTBook2013} for a full formulation}:

\begin{equation*}
\begin{split}
  \infer[\UU\text{Id-F}]{Id_A{M,N} : \UU_i}{A : \UU_i & M,N : A}
  \hspace{50pt}
  \infer[\UU \Pi\text{-F}]{\Pi_{x:A}B : \UU_i}{A : \UU_i & x:A \vdash B : \UU_i}
  \\
  \infer[\UU 1\text{-F}]{1 : \UU_i}{\ctx \mathsf{ctx}}
  \hspace{50pt}
  \infer[\UU 0\text{-F}]{0 : \UU}{\ctx \mathsf{ctx}}
  \hspace{50pt}
  \infer[\UU+\text{-F}]{A + B : \UU}{A:\UU & B:\UU}
\end{split}
\end{equation*}

The addition of universes to ITT solves the problem identified by the running
example.
As the example suggests, these hierarchies increase the expressiveness of ITT.
This is established by identifying a statement that cannot be proven only in
the presence of universes.

\paragraph{example:} 
Jan Smith established that without universes, it is not provable that
$\suc(-) \not = 0$ \cite{Smith84}.  However, in Martin-L\"{o}f's Type Theory, 
it is provable that $n : \mathsf{Nat}$, $\suc(n) \not \equiv 0$\footnote{See
page 86 of Programming in Martin-L\"{o}f's Type Theory \cite{PMLTT}.}.  
In fact, Smith proved that \emph{any} negation of an equivalence cannot be
proven without universes.

\paragraph{exercise:} Show that the operators $\mathsf{fst}$ and $\mathsf{snd}$ 
can be defined from $\mathsf{split}$.

\subsection{Typical Ambiguity}

In the examples above, subscripts on each universe create significant
notational overhead.  Therefore, these indices are elided whenever intent
is obvious.  When implemented with pen and paper, this is called 
\textbf{typical ambiguity}.  Its mechanization in Coq is referred to as 
\textbf{universe polymorphism}.

\subsection{Alternatives to the Hierarchy}

The introduction of an infinite hierarchy of universes complicates the theory.
An uninitiated reader might wonder whether an infinite, cumulative hierarchy is 
really necessary.
This section presents three alternatives.  
The first alternative works, but has some disadvantages.
The other two alternatives have significant problems, demonstrating that the
complexity induced by type universes is essential to a consistent and
sufficiently expressive definition of ITT.

\subsubsection{Large Elimination}

Intensional Type Theory can be consistently formulated without a hierarchy.
The approach, called Large Elimination, rules the correct types into
the theory by hand.  In the case of the running example, the rule would be:

\[
  \infer[\text{LE-If}]{\caseif{M, A, B} \type}
    {M : Bool & A \type & B \type}
\]

Similar rules must be provided for each type formation rule.
The universal approach is preferred because it is less ad hoc \textemdash 
 large elimination requires the addition of new rules for each affected type.

\subsubsection{A Single Universe}

The running example may be addressed without introducing a recursively
defined hierarchy of universes.  One alternative is to replace the $\UU$ rules
above with a single universe.  In this case, the important choice is whether
$\UU : \UU$.

If the universe is not self-inclusive, The formulation problem discussed above 
re-emerges.  For instance, the type $\caseif{M,\UU,\UU \rightarrow \UU}$
is not formable without a recursively defined hierarchy.
The same observation applies at the top of any finite hierarchy.

\subsubsection{The Inconsistent Approach}

An insightful reader might observe that this problem can be
resolved by patching the single universe system with a rule which allows the 
universe to contain itself:

\[
  \infer[\UU\text{-inconsistent}]{\ctx \vdash \UU : \UU}{}
\]

This system allows the formation of $\caseif{M,\UU,\UU \rightarrow \UU}$.
However, it also destroys the consistency of the theory.

\paragraph{Exercise:} Reproduce the Burali-Forte Paradox within a system
equipped with U-cumul-inconsistent\footnote{This formulation of the paradox
is due to Girard 1972, and is referred to as Girard's Paradox.}.


\section{Proof-relevance and Extensionality}\label{sec:extensionality}

Martin-L\"{o}f's Type Theory is significant because it introduces the notion of
\emph{proof relevance}.  Intuitively, this expresses the idea that proofs can be
treated as mathematical objects.

\subsection{The Theorem of Choice}

It is well-known that the Axiom of Choice is independent of the axioms of 
set theory.  However, choice can be derived in ITT.  The derivation provides an 
excellent example of proof relevance in action.

The theorem of choice
states that if $x C y$ is total, then there must exist a function ($f$) which
associates each $x$ with a chosen $y = f(x)$.  We can state this formally
in ITT.

\begin{toc}
$\vdash e : \Pi_{x:A} \Sigma_{y:B} C(x,y) \rightarrow
  \Sigma_{f : A \rightarrow B} . \Pi_{x:A} C(x, f(x))$.
\end{toc}

The proof, provided in \cite{MartinLof80}, involves finding a derivation of:
\begin{equation*}
F :  \Pi_{x:A} \Sigma_{y:B} C(x,y) \vdash \lambda F . <\lambda x . \mathsf{fst} F(x) , \lambda x . \mathsf{snd} F(x)> : \Sigma_{f : A \rightarrow B} . \Pi_{x:A} C(x, f(x))
\end{equation*}

In the proof, $F$ is both an assumption and a mathematical object (namely, a product).
Therefore, the proof may rely upon not only the inhabitation of the type of $F$, but
also $F$ itself.

\paragraph{Note:} In future lectures, banana brackets will be used to recover
a more traditional reading of $F$ by suppressing the ability to use it as a piece
of data in the proof.  For now, the significant observation is that proof
irrelevance can be recovered within ITT.

\subsection{Failure of Extensionality}

Marin-L\"{o}f demonstrated that the Axiom of Extensionality fails in ITT; in ITT,
it is not the case that if $p : Id_A(M, N)$ for closed $M, N, A$, then $M \equiv N : A$.

Extensional Type theory (ETT) addresses the failure of extensionality in ITT by endowing the theory
with the principle of equality of reflection.
Concretely, ETT introduces two new rules which reduce identity to equivalence.
Therefore, all identity paths on a type are the reflexive path.

\begin{equation*}
  \infer[\text{Eq-Refl}]{\ctx \vdash M \equiv N : A}{\ctx \vdash p : Id_A(M,N)}
  \hspace{50pt}
  \infer[\text{UIP}]{\ctx \vdash p \equiv \refl{}(M) : Id_A(M,M)}{p : Id_A(M, N)}
\end{equation*}

The first rule, equality of reflection, states that proof of an identification
is sufficient to show judgemental equality in the type.
The second rule, Uniqueness of Identity Proofs, states that any path is the reflexive path.

Although extensionality does not hold generally for ITT, uniqueness of identity
proofs may be recovered for a large class of types.  

\begin{hedberg}
Any set with decidable identities has collapsed identity sets 
\cite{Hedberg98}\footnote{There is a Coq proof by Nicolai Kraus online: \href{http://www.cs.nott.ac.uk/~ngk/hedberg\_direct.v}{http://www.cs.nott.ac.uk/{\textasciitilde}ngk/hedberg\_direct.v}}.
\end{hedberg}

\subsubsection{ETT vs ITT}

The essential difference between ETT and ITT is the algebraic structure of
types.  ETT reduces all identity paths to reflexivity.  As a result, the path
structure of types in ETT is homotopically discrete.  
ITT admits a much richer path structure on types: two
paths $p : Id_A(A,B)$ and $q : Id_A(B,C)$ may by equal but not trivially equal.

The two other major differences between ETT and ITT are decidability of type
checking and fitness for set-level mathematics.  

The UIP rules introduces proof search as a valid mode of operation for the 
type checker.  Therefore, type checking is not decidable in 
ETT\footnote{Type checking for ITT is decidable}. 
Decidability is not an important criterion for two reasons.  First,
the standard mode of operation in a mechanized ETT (e.g. NuPRL) does not result
in proof search.  Second, type checking in ITT quickly becomes intractable.

A more important secondary distinction between ETT and ITT is fitness for set-
level mathematics.  Types in ETT have the structure of an h-set; 
therefore, set-level mathematics is much nicer in NuPRL than in Coq. Whereas extensionality
and transport come for free in NuPRL, Coq users must induce this structure by
programming in terms of a setoid.  However, the convienance of ETT comes at a cost: 
the path structure of its types in necessarily limited due to Hedberg's Theroem.

Just as proof-relevant mathematics subsumes proof-irrelevant mathematics as a 
special case, the $\infty$-groupoid structure of types in ITT may be
forgotten so that ETT is recovered as a special case.  In fact, this is 
essentially what happens with Setoid in Coq.

\section{Algebraic Structure of Identity Types}\label{sec:structure}

Recall that the induction principle for identity types states that for
$x,y :A $, there exists an identity type $x =_A y$.  Furthermore, proving a
property for these elements and a path $p : x = _A y$ consists of proving the
property in the reflexive case (that is, for $x, x, \refl{x}$).

The full implications of this principle were not understood when it was 
first introduced.  A realization central to Homotopy Type Theory
is that the induction principle for identity types
gives rise to an entire hierarchy of iterated identity types.
That is, due J, we can form the type $p =_{\Id{A}(x,y)} q$ and so on.
Homotopy Type Theory is so-called, in part, because these types form the
same structure as iterated homotopies: that of an $\infty-$groupoid.

Whereas the universes provide a mechanism for reasoning about size in an
iterative fashion, the iterated identities provide an account of dimension.
In the example above, $x$ and $y$ are start and end points.  The first identity 
corresponds to a path between the elements.  Paths between $p,q : \Id{\Id{A}(x,y)}$
are homotopies, or 2-dimensional paths.  Each iteration corresponds to an increase
in dimension\footnote{Dimension is also referred to as homotopy level.}.

Before proceeding with a presentation of the groupoid axioms in terms of 
ITT, it is useful to recall the derivable equivalence relation:

\noindent
\begin{tabular}{l l}
(1) & $id_A(M) := \refl{A}(M) : Id_A(M,M)$ \\
(2) & $p : Id_A(M,N) \vdash p^{-1} : Id_A(N,M)$ \\
(3) & $p : Id_A(M,N), q : id_A(N,P) \vdash p \centerdot q : Id_A(M, P)$ 
\end{tabular}

The second and third are theorems provable by path induction, since
composition and inversion are both defined in terms of J.  Therefore, we may
read identity types propositionally as witnesses of an equivalence, and 
computationally as abstract daata types upon which we may operate.
The two notations for identity types, $x =_A y$ and $\Id{A(x,y)}$, 
typically elucidate the intended reading.
While the former reading comports with more traditional interpretations of
equality, the latter gives rise to the iterated identity types.

\begin{remark}
Despite the correspondence with classical (analytic) homotopy theory,
maps should be thought of synthetically.  As a result, path concatenation
is not defined in terms of function composition; hence, the $\centerdot$
notation.
\end{remark}

\subsection{The Groupoid Laws}

The groupoid laws may be forumated as coherence theorems, each proven by path
induction using J:

\begin{tabular}{l l}
inv-right  & $p \centerdot p_{-1} =_{Id_A(M,M)} \id(M)$ \\
inv-left   & $p^{-1} \centerdot p =_{Id_A(N,N)} \id(N)$ \\
unit-right & $p \centerdot id(N) =_{Id_A(M,N)} p$ \\
unit-left  & $id(M) \centerdot p =_{Id_A(M,N)} p$ \\
assoc      & 
  $(p \centerdot q) \centerdot r =_{Id_A(M, P)} p \centerdot (q \centerdot r)$
\end{tabular}

Before proceeding an explanation of how these are proven, some motivation
may be helpful.  Consider the following diagram for the associativity theorem:

\begin{equation*}
  \begin{tikzcd}
  A \arrow{r} \arrow[bend left]{rr} 
      \arrow[bend left]{rrr}[name=T,below]{}
      \arrow[bend right]{rrr}[name=B]{}
    & B \arrow{r} \arrow[bend right]{rr}
    & C \arrow{r}
    & D
    \arrow[Rightarrow, to path=(T) -- (B)]{}
  \end{tikzcd}
\end{equation*}

This diagram illuminates the weak nature of the path structure: 
associativity holds only because it holds at yet higher type.
Iterated identity types are given structure by this higher coherence.

\begin{thm}
The groupoid laws hold.
\end{thm}

Full proofs are available in chapter 2 of \cite{HoTTBook2013}. We outline
portions of the argument here for the sake of later discussion.
For the first inverse theorem, perform path induction on $p$.  It suffices to consider
that $p = \refl{x}$.  We have by the definition of $\refl{x}^{-1}$ that
$\refl{x} \centerdot \refl{x}^{-1} = \refl{x}$.  The other inverse argument follows
similarly.  The cases for the unit theorems and associativity are similar.  Each follows
by path induction on $p$, considering the case where $p = \refl{x}$.

\subsection{Maps preserve structure}

Given this structure, it is natural to ask whether mappings preserve the 
groupoid structure.  Recall that $\Ap$ preserves identities.

\begin{thm}
If $f : A \rightarrow B$ and $p : M =_A M'$ then $\Ap_f(p) : f M =_B f M'$.
\end{thm}

Mappings preserve not just identity, but the entire groupoid structure.
Proving this requires showing that $ap$ preserves identity, inversion
and composition.  That is,

\begin{thm}
For a function $f : A \rightarrow B$ and paths $p : x =_A y$, $q : y =_A z$
\begin{tabular}{l l}
1) & $\Ap_f(\refl{}(x)) \equiv \refl{}(f(x))$ \\
2) & $\Ap_f(p^{-1}) = ap_f(p)^{-1}$ \\
3) & $\Ap_f(p \centerdot q) =_{\Id{a}(x, z)} ap_f(p) \centerdot ap_f(q)$ \\
\end{tabular}
\end{thm}

\paragraph{Exercse:} Prove that maps preserve functoriality.
A formal proof will be included in a coming revision of these notes.


\subsection{Does Homotopy Type Theory have a Computational Interpretation?}

In the sketch of the groupoid proofs, the general case of $p$ is reduced to
the case of $\refl{x}$.  Currently, this is justified
by a categorical model.  This is not natural or desirable because the 
distinguishing characteristic
of type theory is its computational content characterized by Gentzen's Inversion
Principle.
The model-based justification is insufficient, in part, because it does not provide
a way of \emph{running} HoTT programs.
The constructivity of Homotopy Type Theory is important because Hedberg's Theorem
collapses the dimensional tower developed in this section. However, determining
whether there is a computational interpretation of Homotopy Type theory is
a principle open problem.

%\section{Conclusion}
%These notes have introduced a cumulative hierarchy of universes and an infinite tower of iterated identities.

\bibliographystyle{plain}
\bibliography{hott_references}

\end{document}
