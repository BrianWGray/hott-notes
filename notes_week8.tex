\documentclass[11pt]{article}

\usepackage{proof-dashed}
%%% This file can never be completed.
%%% If you need something but cannot find it,
%%% contact the TA Favonia!

%%%%%%%%%%%%%%%%%%%%%%%%%%%%%%%%%%%%%%%%
% Basic packages
%%%%%%%%%%%%%%%%%%%%%%%%%%%%%%%%%%%%%%%%
\usepackage{amsmath,amsthm,amssymb}
\usepackage{fancyhdr}
\usepackage{mathpartir}
\usepackage{xcolor}
\usepackage{hyperref}
\usepackage{xspace}
\usepackage{comment}
\usepackage{url} % for url in bib entries

%%%%%%%%%%%%%%%%%%%%%%%%%%%%%%%%%%%%%%%%
% Acronyms
%%%%%%%%%%%%%%%%%%%%%%%%%%%%%%%%%%%%%%%%
\usepackage[acronym, shortcuts]{glossaries}

\newacronym{HoTT}{HoTT}{homotopy type theory}
\newacronym{IPL}{IPL}{intuitionistic propositional logic}
\newacronym{TT}{TT}{intuitionistic type theory}
\newacronym{LEM}{LEM}{law of the excluded middle}
\newacronym{ITT}{ITT}{intensional type theory}
\newacronym{ETT}{ETT}{extensional type theory}
\newacronym{NNO}{NNO}{natural numbers object}

%%%%%%%%%%%%%%%%%%%%%%%%%%%%%%%%%%%%%%%%
% Fancy page style
%%%%%%%%%%%%%%%%%%%%%%%%%%%%%%%%%%%%%%%%
\pagestyle{fancy}
\newcommand{\metadata}[2]{
  \lhead{}
  \chead{}
  \rhead{\bfseries Homotopy Type Theory}
  \lfoot{#1}
  \cfoot{#2}
  \rfoot{\thepage}
}
\renewcommand{\headrulewidth}{0.4pt}
\renewcommand{\footrulewidth}{0.4pt}


\newcommand*{\vocab}[1]{\emph{#1}}
\newcommand*{\latin}[1]{\textit{#1}}

%%%%%%%%%%%%%%%%%%%%%%%%%%%%%%%%%%%%%%%%
% Customize list enviroonments
%%%%%%%%%%%%%%%%%%%%%%%%%%%%%%%%%%%%%%%%
% package to customize three basic list environments: enumerate, itemize and description.
\usepackage{enumitem}
\setitemize{noitemsep, topsep=0pt, leftmargin=*}
\setenumerate{noitemsep, topsep=0pt, leftmargin=*}
\setdescription{noitemsep, topsep=0pt, leftmargin=*}

%%%%%%%%%%%%%%%%%%%%%%%%%%%%%%%%%%%%%%%%
% Some really basic macros.
% (Lots of them were stolen from HoTT/Book.)
% See macros.tex in HoTT/book.
%%%%%%%%%%%%%%%%%%%%%%%%%%%%%%%%%%%%%%%%
\newcommand*{\ctx}{\Gamma}
\newcommand{\entails}{\vdash}


\newcommand*{\judgmentfont}[1]{{\normalfont\sffamily #1}}
\newcommand*{\postfixjudgment}[1]{%
  \relax\ifnum\lastnodetype>0\mskip\medmuskip\fi
  \text{\judgmentfont{#1}}%
}
\newcommand*{\prop}{\postfixjudgment{prop}}
\newcommand*{\true}{\postfixjudgment{true}}
\newcommand*{\type}{\postfixjudgment{type}}
\newcommand*{\context}{\postfixjudgment{ctx}}


\newcommand*{\truth}{\top}
\newcommand*{\conj}{\wedge}
\newcommand*{\disj}{\vee}
\newcommand*{\falsehood}{\bot}
\newcommand*{\imp}{\supset}


%%% Judgmental equality
\newcommand{\jdeq}{\equiv}
%%% Definition
\newcommand{\defeq}{\vcentcolon\equiv}
%%% Binary sums
\newcommand{\inlsym}{{\mathsf{inl}}}
\newcommand{\inrsym}{{\mathsf{inr}}}
\newcommand{\inl}{\ensuremath\inlsym\xspace}
\newcommand{\inr}{\ensuremath\inrsym\xspace}
%%% Booleans
\newcommand{\ttsym}{{\mathsf{tt}}}
\newcommand{\ffsym}{{\mathsf{ff}}}
%\newcommand{\ttrue}{\ensuremath\ttsym\xspace}
%\newcommand{\ffalse}{\ensuremath\ffsym\xspace}
%%% Pairs
\newcommand{\pair}{\ensuremath{\mathsf{pair}}\xspace}
\newcommand{\tuple}[2]{(#1,#2)}
\newcommand{\proj}[1]{\ensuremath{\mathsf{pr}_{#1}}\xspace}
\newcommand{\fst}{\ensuremath{\proj1}\xspace}
\newcommand{\snd}{\ensuremath{\proj2}\xspace}
%%% Path concatenation
\newcommand{\concat}{%
  \mathchoice{\mathbin{\raisebox{0.5ex}{$\displaystyle\centerdot$}}}%
  {\mathbin{\raisebox{0.5ex}{$\centerdot$}}}%
  {\mathbin{\raisebox{0.25ex}{$\scriptstyle\,\centerdot\,$}}}%
  {\mathbin{\raisebox{0.1ex}{$\scriptscriptstyle\,\centerdot\,$}}}
}
%%% Transport (covariant)
\newcommand{\trans}[2]{\ensuremath{{#1}_{*}\mathopen{}\left({#2}\right)\mathclose{}}\xspace}
% Natural numbers objects
\newcommand{\Nat}{\mathsf{Nat}}
\newcommand{\rec}{\ensuremath{\mathsf{rec}}\xspace}
% Sequence
\newcommand{\Seq}{\ensuremath{\mathsf{Seq}}\xspace}
% Identity type
\newcommand{\Id}[1]{\ensuremath{\mathsf{Id}_{#1}}\xspace}
% Reflection
\newcommand{\refl}[1]{\ensuremath{\mathsf{refl}_{#1}}\xspace}

% fst,snd,case,id
\renewcommand*{\fst}{\textsf{fst}}
\renewcommand*{\snd}{\textsf{snd}}
\DeclareMathOperator{\case}{\textsf{case}}
\DeclareMathOperator{\caseif}{\textsf{if}}
\DeclareMathOperator{\casesplit}{\textsf{split}}
\DeclareMathOperator{\ttrue}{\textsf{tt}\xspace}
\DeclareMathOperator{\ffalse}{\textsf{ff}\xspace}
\newcommand*{\id}{\textsf{id}}


\setlength{\inferLineSkip}{4pt}

\metadata{Martens and Cavallo}{2013/11/04 and 2013/11/06}

\title{15-819 Homotopy Type Theory\\Lecture Notes}
\author{Evan Cavallo and Chris Martens}
\date{November 4 and 6, 2013}

\begin{document}

\maketitle

\section{Introduction}

% \bibliographystyle{alpha}
% \bibliography{fp}

\section{ITT is a theory of sets}

% XXX Evan should fill in the following, before identity types.
\subsection*{Basic constructs}

(show that $\Pi$, $\Sigma$, etc. respect sethood)

\subsection*{The universe}

We can show that $\universe$ is a set by giving ``codes,'' or abstract
syntax trees, for every type in the universe and showing that they map onto
the natural numbers. % XXX

\subsection*{Identity types}

We can show that $\Id{A}(x,y)$ is a set if $A$ is a set.

Assumption: $A$ is a set, i.e. there is a term $H$ s.t.

\[
H : \Pi{x,y}{:}A.\Pi{p,q}{:}\Id{A}(x,y).\Id{\Id{A}(x,y)}(p,q)
\]

For the sake of making deeply-nested subscripts on identity types more
readable, let's introduce a few definitions:

\newcommand{\idA}{\op{idA}}
\newcommand{\ididA}{\op{ididA}}
\newcommand{\idididA}{\op{idididA}}
\begin{eqnarray*}
\op{idA}(x,y) &:=& \Id{A}(x,y)\\
\op{ididA}(x,y,r,s) &:=& \Id{\idA(x,y)}(r,s) \\
\op{idididA}(x,y,r,s,\alpha,\beta) &:=& \Id{\ididA(x,y,r,s)}(\alpha,\beta)
\end{eqnarray*}

We need to show that for any $x,y$, 
$\Id{A}(x,y)$ is a set, i.e. construct a proof term of type

\[
\Pi{r,s}{:}\idA(x,y).\Pi{\alpha,\beta}{:}\ididA(x,y,r,s).
\idididA(\alpha,\beta)
\]

Assume:
\begin{eqnarray*}
u,v &:& A\\
r,s &:& \idA(u,v)\\
\alpha,\beta &:& \ididA(u,v,r,s)
\end{eqnarray*}

Need to construct a term of type $\idididA(u,v,r,s,\alpha,\beta)$.

First, specialize $H$ to $H'(q) : H(u,v,r,q)$.

% XXX continue here
Exploit functoriality of $H'$

\begin{eqnarray*}
\op{apd}_{H'} &:& \Pi{q,q'}{:}\Id{A}(u,v).
                \Pi{\gamma}{:}\Id{-}(q,q').
                  \Id{-}(\gamma_{*}(H'(q)), H'(q'))\\
\op{apd}_{H'}(r,s,\alpha) &:& \Id{-}(\alpha_{*}(H'(r)), H'(s))\\
\op{apd}_{H'}(r,s,\beta) &:& \Id{-}(\beta_*{}(H'(r)), H'(s))
\end{eqnarray*}

By symmetry and transitivity of identity, we can form a term of type

\[
\Id{-}(\alpha_*(H'(r)), \beta_*(H'(r)))
\]
and so we can get transport in the identity
\[
\Id{-}(H'(r)\cdot \alpha, H'(r)\cdot\beta)
\]
Because the groupid structure tells us we get a cancellation property (?),
this means $\alpha = \beta$

(XXX The above is sloppy; I'll work on it.)

\section{ITT + UA is {\em not} a set theory}

In other words, in homotopy type theory, not all types are sets. In
particular, $\universe$ is a proper groupoid. There are nontrivial paths
between the elements of $\universe$.

As an example, we can demonstrate two paths between the booleans
$\mathbb{2}$, one which is the identity mapping $\op{id}$ taking $\op{tt}$
to $\op{tt}$ and $\op{ff}$ to $\op{ff}$. The other is $\op{not}$, taking
$\op{tt}$ to $\op{ff}$ and $\op{ff}$ to $\op{tt}$.

$\op{not}$ and $\op{id}$ are two paths from $\mathbb{2}$ to $\mathbb{2}$,
and we can demonstrate that they are not identified using univalence.

% XXX do this

\section{$n$-types}

To foreshadow what's to come: we will eventually consider $\op{isSet}(A)$ a
special case of the more general $\op{is-}n\op{-type}(A)$, specifically

% \newcommand{\isntype}[1]{#1}
\newcommand{\isntype}[1]{\op{is-}{#1}\op{-type}}

\begin{tabular}{ccc}
$\op{isSet}(A)$ & becomes & $\isntype{0}(A)$ \\
$\op{isGpd}(A)$ & becomes & $\isntype{1}(A)$ \\
$\op{is2Gpd}(A)$ & becomes & $\isntype{2}(A)$ \\
% $\vdots$ & & $\vdots$
- & - & -
\end{tabular}

The types $A$ for which $\op{is-}n\op{-type}(A)$ holds will be called the
$n$-types. Roughly, it means that ``up a level'' we have a set (the
identities between identities between ... ($n$ times) become identified).

But before we start climbing the ladder upward, let's go the opposite
direction and consider (in some sense) $n=-1, -2$. I.e. what happens if we
{\em take away} structure in the sense of differentiation of identity
proofs.

\section{Proof Irrelevance}

So far, we have taken to heart the idea of {\em proof relevance} and seen
that it can be useful for the {\em evidence for a proposition} to matter,
i.e. to treat the proposition as a type and terms inhabiting that type as
useful, meaningful data. For example, the natural numbers form a type
$\mathbb{N}$, and different ``{\em proofs}'' of $\mathbb{N}$ are different
numbers---so of course we care to differentiate them.

Now we will consider the special case of proof {\em irrelevance}: we can
identify certain propositions for which we {\em do not}
distinguishing its proofs, i.e. we can consider any $M, N : A$ for this type
$A$ to be equivalent. We will call this property $\op{isProp}$
(corresponding to $\op{is-}-1\op{-type}$ in the table above), and formally
we define $\op{isProp}(A)$ to be the type

\[
\Pi{x,y}{:}A.\Id{A}(x,y)
\]

Another word used to describe $A$ with this property is ``subsingleton.''
It is a type with at most one element, up to higher homotopy (i.e. if there
are multiple elements then there are paths between them).

A motivation for considering this type arises in the domain of dependently
typed programming, wherein we want to consider types (propositions) to be
{\em specificatons} for code. For example, consider specifying a function
that takes a (possibly infinite) sequence and returns the first index of
the sequence that contains the element $0$. A type giving this
specification might look like

\[
% \Pi{s{:}\Sigma{t{:}\Nat\to\Nat}}.\ \Sigma{i:\Nat}.s(i)=_{\Nat} 0
\Pi{s{:}\Nat\to\Nat}.\ \Sigma{i:\Nat}.s(i)=_{\Nat} 0
\]



\end{document}
