% arara: pdflatex
% arara: bibtex
% arara: pdflatex
% arara: pdflatex
\documentclass[12pt]{article}

\usepackage{lmodern}
\usepackage{microtype}

% Package for customizing page layout
\usepackage[letterpaper]{geometry}
% \usepackage{fullpage}

%%% This file can never be completed.
%%% If you need something but cannot find it,
%%% contact the TA Favonia!

%%%%%%%%%%%%%%%%%%%%%%%%%%%%%%%%%%%%%%%%
% Basic packages
%%%%%%%%%%%%%%%%%%%%%%%%%%%%%%%%%%%%%%%%
\usepackage{amsmath,amsthm,amssymb}
\usepackage{fancyhdr}
\usepackage{mathpartir}
\usepackage{xcolor}
\usepackage{hyperref}
\usepackage{xspace}
\usepackage{comment}
\usepackage{url} % for url in bib entries

%%%%%%%%%%%%%%%%%%%%%%%%%%%%%%%%%%%%%%%%
% Acronyms
%%%%%%%%%%%%%%%%%%%%%%%%%%%%%%%%%%%%%%%%
\usepackage[acronym, shortcuts]{glossaries}

\newacronym{HoTT}{HoTT}{homotopy type theory}
\newacronym{IPL}{IPL}{intuitionistic propositional logic}
\newacronym{TT}{TT}{intuitionistic type theory}
\newacronym{LEM}{LEM}{law of the excluded middle}
\newacronym{ITT}{ITT}{intensional type theory}
\newacronym{ETT}{ETT}{extensional type theory}
\newacronym{NNO}{NNO}{natural numbers object}

%%%%%%%%%%%%%%%%%%%%%%%%%%%%%%%%%%%%%%%%
% Fancy page style
%%%%%%%%%%%%%%%%%%%%%%%%%%%%%%%%%%%%%%%%
\pagestyle{fancy}
\newcommand{\metadata}[2]{
  \lhead{}
  \chead{}
  \rhead{\bfseries Homotopy Type Theory}
  \lfoot{#1}
  \cfoot{#2}
  \rfoot{\thepage}
}
\renewcommand{\headrulewidth}{0.4pt}
\renewcommand{\footrulewidth}{0.4pt}


\newcommand*{\vocab}[1]{\emph{#1}}
\newcommand*{\latin}[1]{\textit{#1}}

%%%%%%%%%%%%%%%%%%%%%%%%%%%%%%%%%%%%%%%%
% Customize list enviroonments
%%%%%%%%%%%%%%%%%%%%%%%%%%%%%%%%%%%%%%%%
% package to customize three basic list environments: enumerate, itemize and description.
\usepackage{enumitem}
\setitemize{noitemsep, topsep=0pt, leftmargin=*}
\setenumerate{noitemsep, topsep=0pt, leftmargin=*}
\setdescription{noitemsep, topsep=0pt, leftmargin=*}

%%%%%%%%%%%%%%%%%%%%%%%%%%%%%%%%%%%%%%%%
% Some really basic macros.
% (Lots of them were stolen from HoTT/Book.)
% See macros.tex in HoTT/book.
%%%%%%%%%%%%%%%%%%%%%%%%%%%%%%%%%%%%%%%%
\newcommand*{\ctx}{\Gamma}
\newcommand{\entails}{\vdash}


\newcommand*{\judgmentfont}[1]{{\normalfont\sffamily #1}}
\newcommand*{\postfixjudgment}[1]{%
  \relax\ifnum\lastnodetype>0\mskip\medmuskip\fi
  \text{\judgmentfont{#1}}%
}
\newcommand*{\prop}{\postfixjudgment{prop}}
\newcommand*{\true}{\postfixjudgment{true}}
\newcommand*{\type}{\postfixjudgment{type}}
\newcommand*{\context}{\postfixjudgment{ctx}}


\newcommand*{\truth}{\top}
\newcommand*{\conj}{\wedge}
\newcommand*{\disj}{\vee}
\newcommand*{\falsehood}{\bot}
\newcommand*{\imp}{\supset}


%%% Judgmental equality
\newcommand{\jdeq}{\equiv}
%%% Definition
\newcommand{\defeq}{\vcentcolon\equiv}
%%% Binary sums
\newcommand{\inlsym}{{\mathsf{inl}}}
\newcommand{\inrsym}{{\mathsf{inr}}}
\newcommand{\inl}{\ensuremath\inlsym\xspace}
\newcommand{\inr}{\ensuremath\inrsym\xspace}
%%% Booleans
\newcommand{\ttsym}{{\mathsf{tt}}}
\newcommand{\ffsym}{{\mathsf{ff}}}
%\newcommand{\ttrue}{\ensuremath\ttsym\xspace}
%\newcommand{\ffalse}{\ensuremath\ffsym\xspace}
%%% Pairs
\newcommand{\pair}{\ensuremath{\mathsf{pair}}\xspace}
\newcommand{\tuple}[2]{(#1,#2)}
\newcommand{\proj}[1]{\ensuremath{\mathsf{pr}_{#1}}\xspace}
\newcommand{\fst}{\ensuremath{\proj1}\xspace}
\newcommand{\snd}{\ensuremath{\proj2}\xspace}
%%% Path concatenation
\newcommand{\concat}{%
  \mathchoice{\mathbin{\raisebox{0.5ex}{$\displaystyle\centerdot$}}}%
  {\mathbin{\raisebox{0.5ex}{$\centerdot$}}}%
  {\mathbin{\raisebox{0.25ex}{$\scriptstyle\,\centerdot\,$}}}%
  {\mathbin{\raisebox{0.1ex}{$\scriptscriptstyle\,\centerdot\,$}}}
}
%%% Transport (covariant)
\newcommand{\trans}[2]{\ensuremath{{#1}_{*}\mathopen{}\left({#2}\right)\mathclose{}}\xspace}
% Natural numbers objects
\newcommand{\Nat}{\mathsf{Nat}}
\newcommand{\rec}{\ensuremath{\mathsf{rec}}\xspace}
% Sequence
\newcommand{\Seq}{\ensuremath{\mathsf{Seq}}\xspace}
% Identity type
\newcommand{\Id}[1]{\ensuremath{\mathsf{Id}_{#1}}\xspace}
% Reflection
\newcommand{\refl}[1]{\ensuremath{\mathsf{refl}_{#1}}\xspace}

% fst,snd,case,id
\renewcommand*{\fst}{\textsf{fst}}
\renewcommand*{\snd}{\textsf{snd}}
\DeclareMathOperator{\case}{\textsf{case}}
\DeclareMathOperator{\caseif}{\textsf{if}}
\DeclareMathOperator{\casesplit}{\textsf{split}}
\DeclareMathOperator{\ttrue}{\textsf{tt}\xspace}
\DeclareMathOperator{\ffalse}{\textsf{ff}\xspace}
\newcommand*{\id}{\textsf{id}}


\usepackage{proof-dashed}
\usepackage{tikz-cd}
\usepackage[acronym,shortcuts]{glossaries}

\metadata{C.~Newstead}{2013/09/16}

% for url in bib entries
\usepackage{url}

% for theorems, lemmas, etc
\newenvironment{theorem}[1][Theorem.]{\begin{trivlist}\item[\hskip \labelsep {\bfseries #1}]}{\end{trivlist}}
\newenvironment{lemma}[1][Lemma.]{\begin{trivlist}\item[\hskip \labelsep {\bfseries #1}]}{\end{trivlist}}
\newenvironment{corollary}[1][Corollary.]{\begin{trivlist}\item[\hskip \labelsep {\bfseries #1}]}{\end{trivlist}}
\newenvironment{definition}[1][Definition.]{\begin{trivlist}\item[\hskip \labelsep {\bfseries #1}]}{\end{trivlist}}

% \usepackage{parskip}
% \setlength{\parskip}{0.1cm}
% \setlength{\parindent}{0mm}

% package to customize three basic list environments: enumerate, itemize and description.
\usepackage{enumitem}
\setitemize{noitemsep, topsep=0pt, leftmargin=*}
\setenumerate{noitemsep, topsep=0pt, leftmargin=*}
\setdescription{noitemsep, topsep=0pt, leftmargin=*}

\newacronym{DP}{DP}{disjunction property}



\begin{document}

\title{15-819 Homotopy Type Theory\\ Lecture Notes}
\author{Clive Newstead}
\date{September 16, 2013}

\maketitle

\section*{Foreword}

These will undergo substantial revision and expansion in the coming week.

Recall from last time that we can think of the judgement $A \true$ as meaning `$A$ has a proof' and of $A \postfixjudgment{false}$ as `$A$ has a refutation', or equivalently `$\neg A$ has a proof'. These atomic judgements give rise to hypothetical judgements of the form
$$A_1 \true, A_2 \true, \dots, A_n \true \vdash A \true$$
The inference rules of \acl{IPL} then give rise to the structure of a Heyting algebra, called the \emph{Lindenbaum algebra}.

\section{Lindenbaum algebras}

Recall that \acs{IPL} has the structure of a preorder, where we declare $A \le B$ if and only if $A \true \vdash B \true$. Let $T$ be some theory in \acl{IPL} and define a relation $\simeq$ on the propositions in $T$ by
\begin{equation*}
A \simeq B \quad \text{if and only if} \quad A \le B\ \text{and}\ B \le A
\end{equation*}

The fact that $\simeq$ is an equivalence relation follows from the more general fact if $(P, \le)$ is a preorder and a relation $\equiv$ is defined on $P$ by declaring $p \equiv q$ if and only if $p \le q$ and $q \le p$, then $\equiv$ is an equivalence relation on $P$.

\begin{definition}
The \vocab{Lindenbaum algebra} of $T$ is defined to be the collection of $\simeq$-equivalence classes of propositions in $T$. Write $A^* = [A]_{\simeq}$. The ordering on the Lindenbaum algebra is inherited from $\le$.
\end{definition}

%% [COMMENT_CN]: Reword the following theorem
\begin{theorem}
The judgement $\Gamma \vdash A \true$ holds if and only if $\Gamma^* \vdash A^*$ holds in every Heyting algebra.
\end{theorem}
\begin{proof}
Exercise.
\end{proof}

\section{Decidability and stability}

\begin{definition}
$A \prop$ is \emph{decidable} if and only if $A \disj \neg A \true$.
\end{definition}

Decidability is what separates constructuve logic from classical logic: in classical logic, every proposition is decidable (this is precisely the \acl{LEM}), but in constructive logic, this is not so.

A sensible first question to ask might be: `do decidable propositions exist?' Fortunately, the answer is affirmative.
\begin{itemize}
\item $\top$ and $\bot$ are decidable propositions;
\item We would expect $m =_{\mathbb{N}} n$ to be a decidable proposition, where $=_{\mathbb{N}}$ denotes equality on the natural numbers;
\item We would \emph{not} expect $x =_{\mathbb{R}} y$ to be a decidable proposition, where $=_{\mathbb{R}}$ denotes equality on the real numbers, because real numbers are not finite objects.
\end{itemize}

\begin{definition}
$A \prop$ is \emph{stable} if and only if $(\neg \neg A) \imp A \true$.
\end{definition}

Again, in classical logic, every proposition is stable; in fact, the proposition $(\neg \neg A) \imp A \true$ is often taken as an axiom of treatments of classical propositional logic! A natural question to ask now is `do there exist unstable propositions?' Consider the following lemma.
\begin{lemma}
$\neg \neg (A \disj \neg A) \true$
\end{lemma}
\begin{proof}
We must show $\neg (A \disj \neg A) \imp \bot \true$.

Suppose $A \true$. We then have
\begin{equation*}
  \infer{\bot}{
    \infer[{\disj}I_1]{A \vee \neg A \true}{A \true} &
    \neg ( A \vee \neg A ) \true
  }
\end{equation*}

So in fact $\neg A \true$. But then once again
\begin{equation*}
  \infer{\bot}{
    \infer[{\disj} I_2]{A \vee \neg A \true}{\neg A \true} &
    \neg (A \vee \neg A \true)
  }
\end{equation*}

Hence
\begin{equation*}
  \infer[{\imp} I]{\neg (A \disj \neg A) \imp \bot \true}{\neg (A \disj \neg A) \true \vdash \bot}
\end{equation*}
\end{proof}

We can think of this lemma as saying that `the \acl{LEM} is not refutable'. Presuming that there exist undecidable propositions, we obtain the following corollary.

\begin{corollary}
In \acl{IPL}, not every proposition is stable.
\end{corollary}

\section{Disjunction property}
A theory $T$ has the \emph{\ac{DP}} if $T \vdash A \vee B$ implies $T \vdash A$ or $T \vdash B$.

\begin{theorem} \label{thm:dp}
In \acs{IPL}, if $\varnothing \vdash A \disj B \true$ then $\varnothing \vdash A \true$ or $\varnothing \vdash B \true$.
\end{theorem}

\begin{proof}[Na\"{i}ve attempt at proof.]
The idea is to perform induction on all possible derivations $\nabla$ of $\varnothing \vdash A \disj B \true$, with the hope that somewhere along the line we'll find a derivation of $A \true$ or of $B\ true$. Our induction hypothesis is that inside $\nabla$ is enough information to deduce either $\varnothing \vdash A \true$ or $\varnothing \vdash B \true$.

Since $\varnothing \vdash A \vee B \true$ cannot be obtained by assumption or from the rules, ${\conj} I$, ${\imp} I$ or ${\top} I$, we need only consider ${\disj} I_1$, ${\disj} I_2$ and the elimination rules.

If $\varnothing \vdash A \vee B \true$ is obtained from ${\disj} I_1$ then
\begin{equation*}
  \infer[{\disj} I_1]{\varnothing \vdash A \vee B \true}{
    \infer{A \true}{\nabla}
  }
\end{equation*}
so there is a derivation $\nabla$ of $A \true$ and we're done. Likewise if $\varnothing \vdash A \vee B \true$ is obtained from ${\disj} I_2$ then there is a derivation of $B \true$.

If $\varnothing \vdash A \vee B \true$ is obtained from ${\imp} E$ then the deduction takes the form
\begin{equation*}
  \infer[{\imp} E]{\varnothing \vdash A \vee B \true}{
    \infer{\varnothing \vdash C \imp (A \vee B) \true}{\nabla_1} &
    \infer{\varnothing \vdash C \true}{\nabla_2}
  }
\end{equation*}
We (dubiously\footnote{In fact, this `dubious' assumption is true in constructive logic.}) assume that $\vdash C \imp (A \vee B) \true$ must have been derived in some way from $C \true \vdash (A \vee B) \true$. Suppose that this happens and that $\nabla'_1$ is a deduction of $C \true \vdash (A \vee B) \true$. We can then `substitute' $\nabla_2$ for all the occurrences of the assumption $C \true$ appearing in $\nabla'_1$ to obtain a smaller derivation $\nabla_3$ of $\varnothing \vdash A \vee B \true$. Our induction hypothesis then gives us that inside $\nabla_3$ is enough information to deduce $\varnothing \vdash A \true$ or $\varnothing \vdash B \true$.

A similar approach works (we hope) for ${\wedge} E$ and ${\imp} E$, thus giving the result.
\end{proof}

\section{Admissible properties}
The sketch proof of the previous theorem relied on transitivity of $\vdash$; namely, that the following rule is true:
\begin{equation*}
\infer[\mathsf{T}]{\Gamma \vdash B \true}{
  \Gamma, A \true \vdash B \true &
  \Gamma \vdash A \true
}
\end{equation*}
This leads us naturally into a discussion of the structural properties of $\vdash$.

\begin{definition}
A deduction rule is \textit{admissible} (in \acs{IPL}) if nothing changes when it is added to the existing rules of \acs{IPL}.
\end{definition}

To be clear about which logical system we use, we may write $\vdash_{\text{IPL}}$ to denote deduction in \acs{IPL} rather than in some new logical system.

The goal now is to prove that the structural rules for entailment (reflexivity, transitivity, weakening, contraction, exchange) are admissible.

\begin{theorem}
The structural properties of $\vdash_{\text{IPL}}$ are admissible.
\end{theorem}
\begin{proof}
\textsf{R}, \textsf{C}, \textsf{X}: Reflexivity, contraction and exchange are all primitive notions, in that they follow instantly. For instance:
\begin{equation*}
\infer[{\conj} E_1]{\Gamma \vdash A \true}{
  \infer[{\conj} I]{\Gamma \vdash A \conj A \true}{
    \Gamma \vdash A \true
  }
}
\end{equation*}
so if we were to introduce 
\begin{equation*}
\infer[\mathsf{R}]{\Gamma \vdash A \true}{\Gamma \vdash A \true}
\end{equation*}
as a new rule, then nothing would change. (Likewise for contraction and exchange.)

\textsf{W}: For weakening we use the fact that the structural rules are \emph{polymorphic} in $\Gamma$. We can thus prove that weakening is admissible by induction: if the following rules are admissible
\begin{equation*}
\infer{\Gamma, A \true \vdash B_1 \true}{\Gamma \vdash B_2 \true}
\qquad \text{and} \qquad
\infer{\Gamma, A \true \vdash B_1 \true}{\Gamma \vdash B_2 \true}
\end{equation*}
then we obtain
\begin{equation*}
\infer[{\conj} I]{\Gamma, A \true \vdash B_1 \wedge B_2 \true}{
  \infer[\text{Ind}]{\Gamma, A \true \vdash B_1 \true}{
    \infer[{\conj E_1}]{\Gamma \vdash B_1 \true}{\Gamma \vdash B_1 \wedge B_2 \true}
  } & 
  \infer[\text{Ind}]{\Gamma, A \true \vdash B_2 \true}{
    \infer[{\conj E_2}]{\Gamma \vdash B_2 \true}{\Gamma \vdash B_1 \wedge B_2 \true}
  }
}
\end{equation*}
Likewise for the other introduction rules.

\textsf{T}: The admissibility of transitivity is left as an exercise.
\end{proof}




\bibliographystyle{plain}
\bibliography{hott_references}

\end{document}
